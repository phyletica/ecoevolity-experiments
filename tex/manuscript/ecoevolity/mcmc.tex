% \section{Approximating the posterior with MCMC}

% We use Markov chain Monte Carlo (MCMC) algorithms to sample from the joint
% posterior in Equation~\ref{eq:bayesruleexpanded} of the paper.
% To update the divergence model (\divtimemodel) during the chain,
% we use the Gibbs sampling algorithm (Algorithm 8) of \citet{Neal2000}.
% To update each parameter of the model during the MCMC, we use univariate
% Metropolis-Hastings algorithms
% \citep{Metropolis1953,Hastings1970}.
Here, we describe the multivariate Metropolis-Hastings moves that improve
mixing of the MCMC chain when there are strong correlations between divergence
times, effective population sizes, and mutation rates.
The probability of accepting a Metropolis-Hastings proposal is determined
by the product of three terms, the first two of which are the
ratios of the likelihood and prior probability densities of the proposed
state to the current state of the model.
The third term, the Hastings ratio (HR), accounts for any difference in the
probability of the proposed move versus the probability of the move that
would exactly reverse the proposed state back to the current state of the
model.
Below, we detail the Hastings ratios for two of our multivariate moves.

\subsection{\timerootsizemixer proposal}
% \paragraph{\timerootsizemixer proposal}
One case of poor mixing can occur for pairs that diverge long enough ago such
that only a single coalescence occurs within the root for most loci.
In this scenario there is very little information in the character patterns
about the size of the ancestral population, and so the divergence time and root
population size become highly correlated (i.e., an older divergence time and
smaller root size explain the data equally well as a younger divergence time
and larger root size).
We used expectations under the coalescent to design a proposal to better sample
this correlated region of parameter space.
To simplify notation, throughout this section we will use
\sepopsize[\rootpopindex] in place of \epopsize[\rootpopindex] to denote the
effective size of the ancestral population.

\begin{linenomath}
When coalescence of gene lineages is complete within a sampled pair of
populations, only two lineages coalesce within the ancestral population.
In this case, the expected height of the root of a gene tree is equal to
$\divtime\murate + 2\sepopsize[\rootpopindex]\murate$,
in units of time determined by $\murate$ (e.g., expected substitutions per site
if $\murate = 1$).
The purpose of our move is to keep the expected root height of the gene trees
of the proposed state equal to the current state,
\begin{equation}
    \divtime\proposed\murate + 2\sepopsize[\rootpopindex]{}\proposed\murate =
    \divtime\murate + 2\sepopsize[\rootpopindex]\murate.
\end{equation}
The mutation rate cancels, giving us the following relationship to uphold
during our proposal,
\begin{equation}
    \divtime\proposed + 2\sepopsize[\rootpopindex]{}\proposed =
    \divtime + 2\sepopsize[\rootpopindex].
    \label{eq:timerootsizeequality}
\end{equation}
This relationship will allow us to jointly and efficiently explore the space of
\divtime and \sepopsize[\rootpopindex] when there is little information in the
data to tease them apart.
\end{linenomath}

\begin{linenomath}
For population pair $i$, we first draw a uniform random deviate,
$\uniformdeviate \sim \textrm{Uniform}(-\tuningparameter, \tuningparameter)$,
where \tuningparameter is a tuning parameter that can be adjusted to 
improve the acceptance rate of the proposal.
Next, we propose a new value for the effective population size of the
root population
\[
    \sepopsize[\rootpopindex]{}_i\proposed = \sepopsize[\rootpopindex]{}_ie^{\uniformdeviate}.
\]
Now, we use the relationship in Equation~\ref{eq:timerootsizeequality} to
determine the corresponding proposed value for the population divergence time,
\begin{equation}
    \divtime\proposed =
    \divtime + 2\sepopsize[\rootpopindex]{}_i - 2\sepopsize[\rootpopindex]{}_i\proposed.
\end{equation}
The uniform deviate to reverse this move is simply
$\uniformdeviate\proposed = -\uniformdeviate$.
\end{linenomath}


% {\singlespacing
\begin{linenomath}
To get the Hastings ratio for this move, we use the formula of
\citet{Green1995},
\begin{equation}
    \textrm{Hastings ratio} =
    \frac{g\proposed(\uniformdeviate\proposed)}{g(\uniformdeviate)} |\det(J)|,
    \label{eq:greenhastings}
\end{equation}
which is the ratio of the probability of drawing the random deviate that
would reverse the proposed move to the probability of drawing the random
deviate of the proposed move, multiplied by the absolute value of the
determinant of a Jacobian matrix.
% The Jacobian is a matrix of first-order derivatives that describes
% how the move is transforming the variables.
Because the forward and reverse random deviates are uniform,
$\frac{g\proposed(\uniformdeviate\proposed)}{g(\uniformdeviate)} = 1$,
and the Hastings ratio reduces to just the Jacobian term,
\begin{equation}
% \arraycolsep=4pt
\ifdoublespacing{}{
\def\arraystretch{1.5}
}
\begin{split}
    J & = \left[\begin{array}{ccc}
        \frac{\partial \sepopsize[\rootpopindex]{}_i\proposed}{\partial \sepopsize[\rootpopindex]{}_i} &
        \frac{\partial \sepopsize[\rootpopindex]{}_i\proposed}{\partial \divtime} &
        \frac{\partial \sepopsize[\rootpopindex]{}_i\proposed}{\partial \uniformdeviate} \\
        \frac{\partial \divtime\proposed}{\partial \sepopsize[\rootpopindex]{}_i} &
        \frac{\partial \divtime\proposed}{\partial \divtime} &
        \frac{\partial \divtime\proposed}{\partial \uniformdeviate} \\
        \frac{\partial \uniformdeviate\proposed}{\partial \sepopsize[\rootpopindex]{}_i} &
        \frac{\partial \uniformdeviate\proposed}{\partial \divtime} &
        \frac{\partial \uniformdeviate\proposed}{\partial \uniformdeviate}
    \end{array}\right] \\
    & = \left[\begin{array}{ccc}
        e^{\uniformdeviate} &
        0 &
        \sepopsize[\rootpopindex]e^{\uniformdeviate} \\
        2(1 - e^{\uniformdeviate}) &
        1 &
        -2\sepopsize[\rootpopindex]e^{\uniformdeviate} \\
        0 &
        0 &
        -1
    \end{array}\right] \\
    \det(J) & = e^{\uniformdeviate} \left|\begin{array}{cc}
        1 &
        -2\sepopsize[\rootpopindex]e^{\uniformdeviate} \\
        0 &
        -1 
    \end{array}\right| - 0 \left|\begin{array}{cc}
        2(1 - e^{\uniformdeviate}) &
        -2\sepopsize[\rootpopindex]e^{\uniformdeviate} \\
        0 &
        -1
    \end{array}\right| + \sepopsize[\rootpopindex]e^{\uniformdeviate} \left|\begin{array}{cc}
        2(1 - e^{\uniformdeviate}) &
        1 \\
        0 &
        0 \\
    \end{array}\right| \\
    & = -e^{\uniformdeviate} \\
    |\det(J)| & = |-e^{\uniformdeviate}| = e^{\uniformdeviate} = \textrm{Hastings ratio}.
\end{split}
\end{equation}
Notice that the change to \divtime also changes the divergence times
of all the pairs that currently share this divergence time with pair $i$.
So, the efficiency of this move can be hindered when there is a lot of sharing
of divergence times.
However, we can easily extend this move to change the sizes of the ancestral
population of pairs $j, k, \ldots n$ that share their divergence time with
pair $i$.
To do this, we, again, adhere to the relationship in
Equation~\ref{eq:timerootsizeequality}:
\begin{equation}
\begin{split}
    2\sepopsize[\rootpopindex]{}_j\proposed
    & =
    2\sepopsize[\rootpopindex]{}_j + \divtime - \divtime\proposed \\
    \sepopsize[\rootpopindex]{}_j\proposed
    & =
    \sepopsize[\rootpopindex]{}_j + \frac{1}{2}(\divtime - \divtime\proposed) \\
    & =
    \sepopsize[\rootpopindex]{}_j + \frac{1}{2}(\divtime - (\divtime + 2\sepopsize[\rootpopindex]{}_i - 2\sepopsize[\rootpopindex]{}_i\proposed)) \\
    & =
    \sepopsize[\rootpopindex]{}_j + \frac{1}{2}(\divtime - \divtime - 2\sepopsize[\rootpopindex]{}_i + 2\sepopsize[\rootpopindex]{}_i\proposed) \\
    & =
    \sepopsize[\rootpopindex]{}_j - (\sepopsize[\rootpopindex]{}_i - \sepopsize[\rootpopindex]{}_i\proposed) \\
    & =
    \sepopsize[\rootpopindex]{}_j - (\sepopsize[\rootpopindex]{}_i - \sepopsize[\rootpopindex]{}_i e^{\uniformdeviate}) \\
    & =
    \sepopsize[\rootpopindex]{}_j - \sepopsize[\rootpopindex]{}_i(1 - e^{\uniformdeviate}).
\end{split}
\end{equation}
The Jacobian term then becomes
\begin{equation}
% \arraycolsep=4pt
\ifdoublespacing{}{
\def\arraystretch{1.5}
}
\begin{split}
    J & = \left[\begin{array}{cccccccc}
        \frac{\partial \sepopsize[\rootpopindex]{}_i\proposed}{\partial \sepopsize[\rootpopindex]{}_i} &
        \frac{\partial \sepopsize[\rootpopindex]{}_i\proposed}{\partial \divtime} &
        \frac{\partial \sepopsize[\rootpopindex]{}_i\proposed}{\partial \sepopsize[\rootpopindex]{}_j} &
        \frac{\partial \sepopsize[\rootpopindex]{}_i\proposed}{\partial \sepopsize[\rootpopindex]{}_k} &
        \ldots &
        \frac{\partial \sepopsize[\rootpopindex]{}_i\proposed}{\partial \sepopsize[\rootpopindex]{}_{n-1}} &
        \frac{\partial \sepopsize[\rootpopindex]{}_i\proposed}{\partial \sepopsize[\rootpopindex]{}_n} &
        \frac{\partial \sepopsize[\rootpopindex]{}_i\proposed}{\partial \uniformdeviate} \\
        \frac{\partial \divtime\proposed}{\partial \sepopsize[\rootpopindex]{}_i} &
        \frac{\partial \divtime\proposed}{\partial \divtime} &
        \frac{\partial \divtime\proposed}{\partial \sepopsize[\rootpopindex]{}_j} &
        \frac{\partial \divtime\proposed}{\partial \sepopsize[\rootpopindex]{}_k} &
        \ldots &
        \frac{\partial \divtime\proposed}{\partial \sepopsize[\rootpopindex]{}_{n-1}} &
        \frac{\partial \divtime\proposed}{\partial \sepopsize[\rootpopindex]{}_n} &
        \frac{\partial \divtime\proposed}{\partial \uniformdeviate} \\
        \frac{\partial \sepopsize[\rootpopindex]{}_j\proposed}{\partial \sepopsize[\rootpopindex]{}_i} &
        \frac{\partial \sepopsize[\rootpopindex]{}_j\proposed}{\partial \divtime} &
        \frac{\partial \sepopsize[\rootpopindex]{}_j\proposed}{\partial \sepopsize[\rootpopindex]{}_j} &
        \frac{\partial \sepopsize[\rootpopindex]{}_j\proposed}{\partial \sepopsize[\rootpopindex]{}_k} &
        \ldots &
        \frac{\partial \sepopsize[\rootpopindex]{}_j\proposed}{\partial \sepopsize[\rootpopindex]{}_{n-1}} &
        \frac{\partial \sepopsize[\rootpopindex]{}_j\proposed}{\partial \sepopsize[\rootpopindex]{}_n} &
        \frac{\partial \sepopsize[\rootpopindex]{}_j\proposed}{\partial \uniformdeviate} \\
        \frac{\partial \sepopsize[\rootpopindex]{}_k\proposed}{\partial \sepopsize[\rootpopindex]{}_i} &
        \frac{\partial \sepopsize[\rootpopindex]{}_k\proposed}{\partial \divtime} &
        \frac{\partial \sepopsize[\rootpopindex]{}_k\proposed}{\partial \sepopsize[\rootpopindex]{}_j} &
        \frac{\partial \sepopsize[\rootpopindex]{}_k\proposed}{\partial \sepopsize[\rootpopindex]{}_k} &
        \ldots &
        \frac{\partial \sepopsize[\rootpopindex]{}_k\proposed}{\partial \sepopsize[\rootpopindex]{}_{n-1}} &
        \frac{\partial \sepopsize[\rootpopindex]{}_k\proposed}{\partial \sepopsize[\rootpopindex]{}_n} &
        \frac{\partial \sepopsize[\rootpopindex]{}_k\proposed}{\partial \uniformdeviate} \\
        \vdots &
        \vdots &
        \vdots &
        \vdots &
        \ddots &
        \vdots &
        \vdots &
        \vdots \\
        \frac{\partial \sepopsize[\rootpopindex]{}_{n-1}\proposed}{\partial \sepopsize[\rootpopindex]{}_i} &
        \frac{\partial \sepopsize[\rootpopindex]{}_{n-1}\proposed}{\partial \divtime} &
        \frac{\partial \sepopsize[\rootpopindex]{}_{n-1}\proposed}{\partial \sepopsize[\rootpopindex]{}_j} &
        \frac{\partial \sepopsize[\rootpopindex]{}_{n-1}\proposed}{\partial \sepopsize[\rootpopindex]{}_k} &
        \ldots &
        \frac{\partial \sepopsize[\rootpopindex]{}_{n-1}\proposed}{\partial \sepopsize[\rootpopindex]{}_{n-1}} &
        \frac{\partial \sepopsize[\rootpopindex]{}_{n-1}\proposed}{\partial \sepopsize[\rootpopindex]{}_n} &
        \frac{\partial \sepopsize[\rootpopindex]{}_{n-1}\proposed}{\partial \uniformdeviate} \\
        \frac{\partial \sepopsize[\rootpopindex]{}_n\proposed}{\partial \sepopsize[\rootpopindex]{}_i} &
        \frac{\partial \sepopsize[\rootpopindex]{}_n\proposed}{\partial \divtime} &
        \frac{\partial \sepopsize[\rootpopindex]{}_n\proposed}{\partial \sepopsize[\rootpopindex]{}_j} &
        \frac{\partial \sepopsize[\rootpopindex]{}_n\proposed}{\partial \sepopsize[\rootpopindex]{}_k} &
        \ldots &
        \frac{\partial \sepopsize[\rootpopindex]{}_n\proposed}{\partial \sepopsize[\rootpopindex]{}_{n-1}} &
        \frac{\partial \sepopsize[\rootpopindex]{}_n\proposed}{\partial \sepopsize[\rootpopindex]{}_n} &
        \frac{\partial \sepopsize[\rootpopindex]{}_n\proposed}{\partial \uniformdeviate} \\
        \frac{\partial \uniformdeviate\proposed}{\partial \sepopsize[\rootpopindex]{}_i} &
        \frac{\partial \uniformdeviate\proposed}{\partial \divtime} &
        \frac{\partial \uniformdeviate\proposed}{\partial \sepopsize[\rootpopindex]{}_j} &
        \frac{\partial \uniformdeviate\proposed}{\partial \sepopsize[\rootpopindex]{}_k} &
        \ldots &
        \frac{\partial \uniformdeviate\proposed}{\partial \sepopsize[\rootpopindex]{}_{n-1}} &
        \frac{\partial \uniformdeviate\proposed}{\partial \sepopsize[\rootpopindex]{}_n} &
        \frac{\partial \uniformdeviate\proposed}{\partial \uniformdeviate}
    \end{array}\right] \\
    & = \left[\begin{array}{cccccccc}
        e^{\uniformdeviate} &
        0 &
        0 &
        0 &
        \ldots &
        0 &
        0 &
        \sepopsize[\rootpopindex]{}_i e^{\uniformdeviate} \\
        2(1 - e^{\uniformdeviate}) &
        1 &
        0 &
        0 &
        \ldots &
        0 &
        0 &
        -2\sepopsize[\rootpopindex]{}_i e^{\uniformdeviate} \\
        e^{\uniformdeviate} - 1 &
        0 &
        1 &
        0 &
        \ldots &
        0 &
        0 &
        \sepopsize[\rootpopindex]{}_i e^{\uniformdeviate} \\
        e^{\uniformdeviate} - 1 &
        0 &
        0 &
        1 &
        \ddots &
        0 &
        0 &
        \sepopsize[\rootpopindex]{}_i e^{\uniformdeviate} \\
        \vdots &
        \vdots &
        \vdots &
        \ddots &
        \ddots &
        \ddots &
        \vdots &
        \vdots \\
        e^{\uniformdeviate} - 1 &
        0 &
        0 &
        0 &
        \ddots &
        1 &
        0 &
        \sepopsize[\rootpopindex]{}_i e^{\uniformdeviate} \\
        e^{\uniformdeviate} - 1 &
        0 &
        0 &
        0 &
        \ldots &
        0 &
        1 &
        \sepopsize[\rootpopindex]{}_i e^{\uniformdeviate} \\
        0 &
        0 &
        0 &
        0 &
        \ldots &
        0 &
        0 &
        -1
    \end{array}\right] \\
    \det(J) & = -e^{\uniformdeviate} \\
    |\det(J)| & = |-e^{\uniformdeviate}| = e^{\uniformdeviate} = \textrm{Hastings ratio}.
\end{split}
\end{equation}
\end{linenomath}
% }

\subsection{\timesizeratemixer proposal}
% \paragraph{\timesizeratemixer proposal}
This proposal is designed to improve mixing of the MCMC chain when there are
strong posterior correlations among divergence time, population size, and
mutation rate parameters.
It does so by jointly scaling these parameters according to the direction
(positive or negative) of the posterior correlations we often observed when
analyzing simulated data.
The divergence time was often positively correlated with the effective sizes of
the descendant populations, and negatively correlated with the mutation rate
and effective population size of the ancestral population.

\begin{linenomath}
For a given divergence time, \divtime[i],
we first draw a random uniform deviate,
$\uniformdeviate \sim \textrm{Uniform}(-\tuningparameter, \tuningparameter)$,
where \tuningparameter is, again, a tuning parameter to adjust the proposal's
acceptance rate.
We use this random deviate to propose a new value for the divergence time,
\[
    \divtime[i]\proposed = \divtime[i] e^{\uniformdeviate}.
\]
Next, we visit each population pair that is associated with this
divergence time, and propose the following updates to the
pair's parameters, if they are being estimated (i.e., not fixed):
\begin{equation}
    \begin{split}
        \epopsize[\rootpopindex]{}\proposed & = \epopsize[\rootpopindex]{} e^{-\uniformdeviate} \\
        \epopsize[\descendantpopindex{1}]{}\proposed & = \epopsize[\descendantpopindex{1}]{} e^{\uniformdeviate} \\
        \epopsize[\descendantpopindex{2}]{}\proposed & = \epopsize[\descendantpopindex{1}]{} e^{\uniformdeviate} \\
        \murate\proposed & = \murate e^{-\uniformdeviate}.
    \end{split}
\end{equation}
When doing so, we keep track of the total number of parameters
that have been updated, denoted as $n$,
and how many of these were scaled by
$e^{\uniformdeviate}$, denoted $m$;
the remaining $n-m$ parameters were scaled by $e^{-\uniformdeviate}$.
\end{linenomath}

% {\singlespacing
\begin{linenomath}
Given $n$ and $m$, we can again use Green's \citeyear{Green1995} formula
(Equation~\ref{eq:greenhastings} above) to determine the Hastings ratio for
this proposal.
Once again, 
$\frac{g\proposed(\uniformdeviate\proposed)}{g(\uniformdeviate)} = 1$,
because the random deviates are uniform.
Using
$\parameter[1], \ldots, \parameter[m]$
and
$\parameter[m+1], \ldots, \parameter[n]$
to denote the parameters that
have been scaled by
$e^{\uniformdeviate}$
and
$e^{-\uniformdeviate}$, respectively,
the Jacobian term is
\begin{equation}
% \arraycolsep=4pt
\ifdoublespacing{}{
\def\arraystretch{1.5}
}
\begin{split}
    J & = \left[\begin{array}{ccccccccccc}
        \frac{\partial \parameter[1]\proposed}{\partial \parameter[1]} &
        \frac{\partial \parameter[1]\proposed}{\partial \parameter[2]} &
        \ldots &
        \frac{\partial \parameter[1]\proposed}{\partial \parameter[m-1]} &
        \frac{\partial \parameter[1]\proposed}{\partial \parameter[m]} &
        \frac{\partial \parameter[1]\proposed}{\partial \parameter[m+1]} &
        \frac{\partial \parameter[1]\proposed}{\partial \parameter[m+2]} &
        \ldots &
        \frac{\partial \parameter[1]\proposed}{\partial \parameter[n-1]} &
        \frac{\partial \parameter[1]\proposed}{\partial \parameter[n]} &
        \frac{\partial \parameter[1]\proposed}{\partial \uniformdeviate} \\
        \frac{\partial \parameter[2]\proposed}{\partial \parameter[1]} &
        \frac{\partial \parameter[2]\proposed}{\partial \parameter[2]} &
        \ldots &
        \frac{\partial \parameter[2]\proposed}{\partial \parameter[m-1]} &
        \frac{\partial \parameter[2]\proposed}{\partial \parameter[m]} &
        \frac{\partial \parameter[2]\proposed}{\partial \parameter[m+1]} &
        \frac{\partial \parameter[2]\proposed}{\partial \parameter[m+2]} &
        \ldots &
        \frac{\partial \parameter[2]\proposed}{\partial \parameter[n-1]} &
        \frac{\partial \parameter[2]\proposed}{\partial \parameter[n]} &
        \frac{\partial \parameter[2]\proposed}{\partial \uniformdeviate} \\
        \vdots &
        \vdots &
        \ddots &
        \vdots &
        \vdots &
        \vdots &
        \vdots &
        \ddots &
        \vdots &
        \vdots &
        \vdots \\
        \frac{\partial \parameter[m-1]\proposed}{\partial \parameter[1]} &
        \frac{\partial \parameter[m-1]\proposed}{\partial \parameter[2]} &
        \ldots &
        \frac{\partial \parameter[m-1]\proposed}{\partial \parameter[m-1]} &
        \frac{\partial \parameter[m-1]\proposed}{\partial \parameter[m]} &
        \frac{\partial \parameter[m-1]\proposed}{\partial \parameter[m+1]} &
        \frac{\partial \parameter[m-1]\proposed}{\partial \parameter[m+2]} &
        \ldots &
        \frac{\partial \parameter[m-1]\proposed}{\partial \parameter[n-1]} &
        \frac{\partial \parameter[m-1]\proposed}{\partial \parameter[n]} &
        \frac{\partial \parameter[m-1]\proposed}{\partial \uniformdeviate} \\
        \frac{\partial \parameter[m]\proposed}{\partial \parameter[1]} &
        \frac{\partial \parameter[m]\proposed}{\partial \parameter[2]} &
        \ldots &
        \frac{\partial \parameter[m]\proposed}{\partial \parameter[m-1]} &
        \frac{\partial \parameter[m]\proposed}{\partial \parameter[m]} &
        \frac{\partial \parameter[m]\proposed}{\partial \parameter[m+1]} &
        \frac{\partial \parameter[m]\proposed}{\partial \parameter[m+2]} &
        \ldots &
        \frac{\partial \parameter[m]\proposed}{\partial \parameter[n-1]} &
        \frac{\partial \parameter[m]\proposed}{\partial \parameter[n]} &
        \frac{\partial \parameter[m]\proposed}{\partial \uniformdeviate} \\
        \frac{\partial \parameter[m+1]\proposed}{\partial \parameter[1]} &
        \frac{\partial \parameter[m+1]\proposed}{\partial \parameter[2]} &
        \ldots &
        \frac{\partial \parameter[m+1]\proposed}{\partial \parameter[m-1]} &
        \frac{\partial \parameter[m+1]\proposed}{\partial \parameter[m]} &
        \frac{\partial \parameter[m+1]\proposed}{\partial \parameter[m+1]} &
        \frac{\partial \parameter[m+1]\proposed}{\partial \parameter[m+2]} &
        \ldots &
        \frac{\partial \parameter[m+1]\proposed}{\partial \parameter[n-1]} &
        \frac{\partial \parameter[m+1]\proposed}{\partial \parameter[n]} &
        \frac{\partial \parameter[m+1]\proposed}{\partial \uniformdeviate} \\
        \frac{\partial \parameter[m+2]\proposed}{\partial \parameter[1]} &
        \frac{\partial \parameter[m+2]\proposed}{\partial \parameter[2]} &
        \ldots &
        \frac{\partial \parameter[m+2]\proposed}{\partial \parameter[m-1]} &
        \frac{\partial \parameter[m+2]\proposed}{\partial \parameter[m]} &
        \frac{\partial \parameter[m+2]\proposed}{\partial \parameter[m+1]} &
        \frac{\partial \parameter[m+2]\proposed}{\partial \parameter[m+2]} &
        \ldots &
        \frac{\partial \parameter[m+2]\proposed}{\partial \parameter[n-1]} &
        \frac{\partial \parameter[m+2]\proposed}{\partial \parameter[n]} &
        \frac{\partial \parameter[m+2]\proposed}{\partial \uniformdeviate} \\
        \vdots &
        \vdots &
        \ddots &
        \vdots &
        \vdots &
        \vdots &
        \vdots &
        \ddots &
        \vdots &
        \vdots &
        \vdots \\
        \frac{\partial \parameter[n-1]\proposed}{\partial \parameter[1]} &
        \frac{\partial \parameter[n-1]\proposed}{\partial \parameter[2]} &
        \ldots &
        \frac{\partial \parameter[n-1]\proposed}{\partial \parameter[m-1]} &
        \frac{\partial \parameter[n-1]\proposed}{\partial \parameter[m]} &
        \frac{\partial \parameter[n-1]\proposed}{\partial \parameter[m+1]} &
        \frac{\partial \parameter[n-1]\proposed}{\partial \parameter[m+2]} &
        \ldots &
        \frac{\partial \parameter[n-1]\proposed}{\partial \parameter[n-1]} &
        \frac{\partial \parameter[n-1]\proposed}{\partial \parameter[n]} &
        \frac{\partial \parameter[n-1]\proposed}{\partial \uniformdeviate} \\
        \frac{\partial \parameter[n]\proposed}{\partial \parameter[1]} &
        \frac{\partial \parameter[n]\proposed}{\partial \parameter[2]} &
        \ldots &
        \frac{\partial \parameter[n]\proposed}{\partial \parameter[m-1]} &
        \frac{\partial \parameter[n]\proposed}{\partial \parameter[m]} &
        \frac{\partial \parameter[n]\proposed}{\partial \parameter[m+1]} &
        \frac{\partial \parameter[n]\proposed}{\partial \parameter[m+2]} &
        \ldots &
        \frac{\partial \parameter[n]\proposed}{\partial \parameter[n-1]} &
        \frac{\partial \parameter[n]\proposed}{\partial \parameter[n]} &
        \frac{\partial \parameter[n]\proposed}{\partial \uniformdeviate} \\
        \frac{\partial \parameter[\uniformdeviate]\proposed}{\partial \parameter[1]} &
        \frac{\partial \parameter[\uniformdeviate]\proposed}{\partial \parameter[2]} &
        \ldots &
        \frac{\partial \parameter[\uniformdeviate]\proposed}{\partial \parameter[m-1]} &
        \frac{\partial \parameter[\uniformdeviate]\proposed}{\partial \parameter[m]} &
        \frac{\partial \parameter[\uniformdeviate]\proposed}{\partial \parameter[m+1]} &
        \frac{\partial \parameter[\uniformdeviate]\proposed}{\partial \parameter[m+2]} &
        \ldots &
        \frac{\partial \parameter[\uniformdeviate]\proposed}{\partial \parameter[n-1]} &
        \frac{\partial \parameter[\uniformdeviate]\proposed}{\partial \parameter[n]} &
        \frac{\partial \parameter[\uniformdeviate]\proposed}{\partial \uniformdeviate} \\
    \end{array}\right] \\
    & = \left[\begin{array}{ccccccccccc}
        e^{\uniformdeviate} &
        0 &
        \ldots &
        0 &
        0 &
        0 &
        0 &
        \ldots &
        0 &
        0 &
        \parameter[1]\proposed e^{\uniformdeviate} \\
        0 &
        \ddots &
        \ddots &
        0 &
        0 &
        0 &
        0 &
        \ldots &
        0 &
        0 &
        \parameter[1]\proposed e^{\uniformdeviate} \\
        \vdots &
        \ddots &
        \ddots &
        \ddots &
        \vdots &
        \vdots &
        \vdots &
        \ddots &
        \vdots &
        \vdots &
        \vdots \\
        0 &
        0 &
        \ddots &
        \ddots &
        0 &
        0 &
        0 &
        \ldots &
        0 &
        0 &
        \parameter[m]\proposed e^{\uniformdeviate} \\
        0 &
        0 &
        \ldots &
        0 &
        e^{\uniformdeviate} &
        0 &
        0 &
        \ldots &
        0 &
        0 &
        \parameter[m]\proposed e^{\uniformdeviate} \\
        0 &
        0 &
        \ldots &
        0 &
        0 &
        e^{-\uniformdeviate} &
        0 &
        \ldots &
        0 &
        0 &
        \parameter[m]\proposed e^{-\uniformdeviate} \\
        0 &
        0 &
        \ldots &
        0 &
        0 &
        0 &
        \ddots &
        \ddots &
        0 &
        0 &
        \parameter[m]\proposed e^{-\uniformdeviate} \\
        \vdots &
        \vdots &
        \ddots &
        \vdots &
        \vdots &
        \vdots &
        \ddots &
        \ddots &
        \ddots &
        \vdots &
        \vdots \\
        0 &
        0 &
        \ldots &
        0 &
        0 &
        0 &
        0 &
        \ddots &
        \ddots &
        0 &
        \parameter[n]\proposed e^{-\uniformdeviate} \\
        0 &
        0 &
        \ldots &
        0 &
        0 &
        0 &
        0 &
        \ldots &
        0 &
        e^{-\uniformdeviate} &
        \parameter[n]\proposed e^{-\uniformdeviate} \\
        0 &
        0 &
        \ldots &
        0 &
        0 &
        0 &
        0 &
        \ldots &
        0 &
        0 &
        -1 \\
    \end{array}\right] \\
    \det(J) & = -e^{\uniformdeviate{}m}e^{-\uniformdeviate{}(n - m)} \\
    & = -e^{2\uniformdeviate{}m - \uniformdeviate{}n} \\
    |\det(J)| & = e^{\uniformdeviate{}(2m - n)} = \textrm{Hastings ratio}.
\end{split}
\end{equation}
\end{linenomath}
% }
