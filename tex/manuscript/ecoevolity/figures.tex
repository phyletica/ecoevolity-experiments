\mFigure{../../images/div-cartoon/div-cartoon.pdf}{
    A cartoon depiction of the inference problem for three pairs of
    insular lizard populations.
    Three ancestral species of lizards co-occured on a paleo-island that was
    fragmented into two islands by a rise in sea levels at \divtime[1].
    The island fragmentation caused the second and third (from the top)
    lineages to co-diverge; the first lineage diverged later (at \divtime[2])
    via over-water dispersal.
    The five possible divergence models are shown to the right, with the
    correct model indicated.
    The divergence-time parameters (\divtime[1] and \divtime[2])
    and the pair-specific divergence times (\comparisondivtime[1],
    \comparisondivtime[2], and \comparisondivtime[3])
    are shown.
    The third population pair shows the notation used in the text for the
    biallelic character data
    ($\leafallelecounts, \leafredallelecounts = (\allelecount[1],
    \redallelecount[1]), (\allelecount[2], \redallelecount[2])$)
    and effective sizes of the ancestral
    (\epopsize[\rootpopindex{}]) and descendant
    (\epopsize[\descendantpopindex{1}] and \epopsize[\descendantpopindex{2}])
    populations.
    The lizard silhouette for the middle pair is from pixabay.com, and the
    other two are from phylopic.org; all were licensed under the Creative
    Commons (CC0) 1.0 Universal Public Domain Dedication.
}{fig:divCartoon}

\mFigure{../../images/ecoevolity-dag-model.pdf}{
    A directed graph representation of the model implemented in
    \ecoevolity.
    Each $p_i$ represents a parameter of a prior for which the prior
    distribution and value(s) of the parameter(s) can be choosen by the
    investigator.
    \thought{This is so complex I don't know if it's worth it.}
}{fig:dag}

\mFigure{../../../simulations/validation/plots/div-time-scatter.pdf}{
    The accuracy and precision of divergence time estimates, in units of
    expected subsitutions per site, when data are simulated and analyzed
    under the same model
    (i.e., no model misspecification).
    \validationfiguregriddescription
    Each plotted circle and associated error bars represent the posterior mean
    and 95\% credibility interval for the time that a pair of populations
    diverged.
    Each plot consists of 1500 estimates---500 simulated \datasets, each with
    three pairs of populations.
    \accuracyscatterplotannotations{\comparisondivtime{}}
    \weusedmatplotlib
}{fig:valdivtimes}

\mFigure{../../../simulations/validation/plots/100k-sites-est-vs-true-prob-nevent-1.pdf}{
    Assessing frequentist behavior of divergence-model posterior probabilities
    when there is no model misspecification.
    20,500 \datasets were simulated and analyzed under the same model and
    assigned to bins of width 0.2 based on the estimated posterior
    probability of a single, shared divergence event.
    The mean posterior probability of each bin is plotted agains the proportion
    of \datasets in the bin for which a single, shared divergence is the
    true model.
    The number of \datasets within each bin is provided next to the
    corresponding plotted point.
    The left plot shows the results when all characters are analyzed, and the
    right plot shows the results when only the variable characters are
    analyzed.
    All simulated \datasets had three pairs of populations, each with 100,000
    characters.
    \weusedmatplotlib
}{fig:valpostprobs}

\mFigure{../../../simulations/validation/plots/nevents.pdf}{
    The ability of the new method to estimate the number of divergence events
    when data are simulated and analyzed
    under the same model
    (i.e., no model misspecification).
    \validationfiguregriddescription
    Each plot shows the results of the analyses of 500 simulated \datasets,
    each with three population pairs;
    the number of \datasets that fall within each possible cell
    of true versus estimated numbers of events is shown, and cells with
    more \datasets are shaded darker.
    \neventplotannotations
    \weusedmatplotlib
}{fig:valnevents}


\mFigure{../../../simulations/validation/plots/linkage-500k-div-time-scatter.pdf}{
    Assessing the affect of linked sites on the
    the accuracy and precision of divergence time estimates (in units of
    expected subsitutions per site).
    The columns, from left to right, show the results when loci are simulated
    with 100, 500, and 1000 linked sites.
    For each simulated \dataset, each of three population pairs has
    500,000 sites total.
    The rows show the results when (top) all sites, (middle) all variable
    sites, and (bottom) at most one variable site per locus are analyzed.
    \accuracyscatterplotannotations{\comparisondivtime{}}
    \weusedmatplotlib
}{fig:linkagedivtimes500k}

\mFigure{../../../simulations/validation/plots/linkage-500k-nevents.pdf}{
    Assessing the affect of linked sites on the ability of the new method to
    estimate the number of divergence events.
    The columns, from left to right, show the results when loci are simulated
    with 100, 500, and 1000 linked sites.
    For each simulated \dataset, each of three population pairs has
    500,000 sites total.
    The rows show the results when (top) all sites, (middle) all variable
    sites, and (bottom) at most one variable site per locus are analyzed.
    \neventplotannotations
    \weusedmatplotlib
}{fig:linkagenevents500k}

\mFigure{../../../simulations/validation/plots/linkage-100-100k-sites-est-vs-true-prob-nevent-1.pdf}{
    Assessing the affect of linked sites on the frequentist behavior of
    divergence-model posterior probabilities.
    10,500 \datasets were simulated such that each of three population pairs
    has 1000 loci, each with 100 linked sites (100,000 sites total).
    Each simulated \dataset is assigned to a bin of width 0.2 based on the
    estimated posterior probability of a single, shared divergence event.
    The mean posterior probability of each bin is plotted agains the proportion
    of \datasets in the bin for which a single, shared divergence is the
    true model.
    The number of \datasets within each bin is provided next to the
    corresponding plotted point.
    The plots show the results when (left) all characters, (middle) all
    variable characters, and (right) at most one variable character per locus
    is analyzed.
    \weusedmatplotlib
}{fig:linkagepostprobs}


\mFigure{../../../simulations/validation/plots/missing-data-div-time-scatter.pdf}{
    Assessing the affect of missing data on the
    the accuracy and precision of divergence time estimates (in units of
    expected subsitutions per site).
    The columns, from left to right, show the results when each simulated
    500,000-character matrix has approximately 0\%, 10\%, 25\%, and 50\%
    missing cells.
    The rows show the results when (top) all sites and (bottom) only variable
    sites are analyzed.
    \accuracyscatterplotannotations{\comparisondivtime{}}
    All simulated \datasets had three pairs of populations.
    \weusedmatplotlib
}{fig:missingdivtimes}

\mFigure{../../../simulations/validation/plots/missing-data-nevents.pdf}{
    Assessing the affect of missing data on the ability of the new method to
    estimate the number of divergence events.
    The columns, from left to right, show the results when each simulated
    500,000-character matrix has approximately 0\%, 10\%, 25\%, and 50\%
    missing cells.
    The rows show the results when (top) all sites and (bottom) only variable
    sites are analyzed.
    \neventplotannotations
    All simulated \datasets had three pairs of populations.
    \weusedmatplotlib
}{fig:missingnevents}


\mFigure{../../../simulations/validation/plots/filtered-data-div-time-scatter.pdf}{
    Assessing the affect of an acquisition bias against rare allele patterns on
    the accuracy and precision of divergence time estimates (in units of
    expected subsitutions per site).
    The columns, from left to right, show the results when each simulated
    500,000-character \dataset has a probability of 100\%, 80\%, 60\%, and 40\%
    of sampling each simulated singleton pattern.
    I.e., each character matrix analyzed in the far right column is missing
    approximately 60\% of characters where all but one gene copy has the same
    allele.
    The rows show the results when (top) all sites and (bottom) only variable
    sites are analyzed.
    \accuracyscatterplotannotations{\comparisondivtime{}}
    All simulated \datasets had three pairs of populations.
    \weusedmatplotlib
}{fig:filtereddivtimes}

\mFigure{../../../simulations/validation/plots/filtered-data-nevents.pdf}{
    Assessing the affect of an acquisition bias against rare allele patterns
    on the ability of the new method to estimate the number of divergence
    events.
    The columns, from left to right, show the results when each simulated
    500,000-character \dataset has a probability of 100\%, 80\%, 60\%, and 40\%
    of sampling each simulated singleton pattern.
    I.e., each character matrix analyzed in the far right column is missing
    approximately 60\% of characters where all but one gene copy has the same
    allele.
    The rows show the results when (top) all sites and (bottom) only variable
    sites are analyzed.
    \neventplotannotations
    All simulated \datasets had three pairs of populations.
    \weusedmatplotlib
}{fig:filterednevents}


\mFigure{../../../bake-off/plots/divergence-time-scatter.pdf}{
    Comparing the accuracy and precision of divergence-time estimates between
    (left) the new full-likelihood Bayesian method, \ecoevolity, and (right)
    the approximate-likelihood Bayesian method, \dppmsbayes.
    Each plotted circle and associated error bars represent the posterior mean
    and 95\% credibility interval for the time that a pair of populations
    diverged.
    Each plot consists of 1500 estimates---500 simulated \datasets, each with
    three pairs of populations.
    The simulated character matrix for each population pair consisted of 200 loci,
    each with 200 linked sites (40,000 characters total).
    \accuracyscatterplotannotations{\comparisondivtime{}}
    \weusedmatplotlib
}{fig:bakeoffdivtimes}

\mFigure{../../../bake-off/plots/nevents.pdf}{
    Comparing the ability to estimate the number of divergence events between
    (left) the new full-likelihood Bayesian method, \ecoevolity, and (right)
    the approximate-likelihood Bayesian method, \dppmsbayes.
    Each plot shows the results of the analyses of 500 simulated \datasets;
    the number of \datasets that fall within each possible cell
    of true versus estimated numbers of events is shown, and cells with
    more \datasets are shaded darker.
    Each simulated \dataset contained three pairs of populations, and the
    simulated character matrix for each pair consisted of 200 loci, each with
    200 linked sites (40,000 characters total).
    \neventplotannotations
    \weusedmatplotlib
}{fig:bakeoffnevents}


\mFigure{../../../gekko-results/sumevents.pdf}{
    The prior (light bars) and approximated posterior (dark bars) probabilities
    of the number of divergence events across \spp{Gekko} pairs of populations,
    under eight different combinations of priors on the divergence times (rows)
    and the concentration parameter of the Dirichlet process (columns).
    For these analyses, constant characters were included, and all characters
    with more than two alleles were recoded as biallelic.
    The Bayes factor for each number of divergence times is given above the
    corresponding bars.
    Each Bayes factor compares the corresponding number of events 
    to all other possible numbers of divergence events.
    \weusedggplot
}{fig:gekkonevents}

\mFigure{../../../gekko-results/sumtimes.pdf}{
    The approximate marginal posterior densities of divergence times for each
    \spp{Gekko} pair of populations,
    under eight different combinations of priors on the divergence times (rows)
    and the concentration parameter of the Dirichlet process (columns).
    For these analyses, constant characters were included, and all characters
    with more than two alleles were recoded as biallelic.
    \weusedggridges
}{fig:gekkodivtimes}
