\embedWidthFigure{1.0}{../../images/ecoevolity-dag-model.pdf}{
    A directed graph representation of the model implemented in
    \ecoevolity.
    Each constant node represents the parameters of a prior for which the prior
    distribution and values of the parameters can be choosen by the
    investigator.
    % \thought{This is so complex I don't know if it's worth it.}
}{fig:dag}


\siFigure{../../../simulations/validation/plots/leaf-pop-size-scatter.pdf}{
    The accuracy and precision of estimates of the descendant (leaf) population
    sizes (scaled by the mutation rate), when data are simulated and analyzed
    under the same model (i.e., no model misspecification).
    The columns show the
    results from different distributions on the relative effective size of the
    ancestral population, decreasing in variance from left to right.
    For the first two and last two rows, the simulated character matrix for
    each population had 500,000 and 100,000 characters, respectively.
    The first and third rows show the results of analyses using all characters,
    whereas the second and fourth rows show the results when only variable
    characters are used.
    Each plotted circle and associated error bars represent the posterior mean
    and 95\% credible interval.
    Each plot consists of 3000 estimates---500 simulated \datasets, each with
    three pairs of populations.
    \accuracyscatterplotannotations{\epopsize{}\murate{}}
    \weusedmatplotlib
}{fig:valleafsizes}

\siFigure{../../../simulations/validation/plots/root-pop-size-scatter.pdf}{
    The accuracy and precision of estimates of the ancestral (root) population
    size (scaled by the mutation rate), when data are simulated and analyzed
    under the same model (i.e., no model misspecification).
    The columns show the
    results from different distributions on the relative effective size of the
    ancestral population, decreasing in variance from left to right.
    For the first two and last two rows, the simulated character matrix for
    each population had 500,000 and 100,000 characters, respectively.
    The first and third rows show the results of analyses using all characters,
    whereas the second and fourth rows show the results when only variable
    characters are used.
    Each plotted circle and associated error bars represent the posterior mean
    and 95\% credible interval.
    Each plot consists of 1500 estimates---500 simulated \datasets, each with
    three pairs of populations.
    \accuracyscatterplotannotations{\epopsize{}\murate{}}
    \weusedmatplotlib
}{fig:valrootsizes}

\siFigure{\ifuserasterizedplotsinsi{../../../simulations/validation/plots/models-compressed.pdf}{../../../simulations/validation/plots/models.pdf}}{
    \jroedit{}{The ability of the new method to estimate the model of
    divergence when data are simulated and analyzed under the same model (i.e.,
    no model misspecification).
    \validationfiguregriddescription
    Each plot shows the results of the analyses of 500 simulated \datasets,
    each with three population pairs;
    the number of \datasets that fall within each possible cell
    of true versus estimated model is shown, and cells with
    more \datasets are shaded darker.
    \modelplotannotations
    \weusedmatplotlib
    }
}{fig:valmodels}

\siFigure{../../../simulations/validation/plots/number-of-variable-sites-100k-histograms.pdf}{
    The number of variable characters in the simulated \datasets with 100,000
    unlinked and unfiltered characters per pair of populations.
    500 \datasets were simulated for each setting on the relative size of the
    ancestral population (indicated above each plot).
    The mean and range across the 500 \datasets is indicated
    in the upper right corner of each plot.
    \weusedmatplotlib
}{fig:valnumberofvariablesites100k}

\siFigure{../../../simulations/validation/plots/number-of-variable-sites-500k-histograms.pdf}{
    The number of variable characters in the simulated \datasets with 500,000 
    unlinked and unfiltered characters per pair of populations.
    500 \datasets were simulated for each setting on the relative size of the
    ancestral population (indicated above each plot).
    The mean and range across the 500 \datasets is indicated
    in the upper right corner of each plot.
    \weusedmatplotlib
}{fig:valnumberofvariablesites500k}

\siFigure{../../../simulations/validation/plots/psrf-ln-likelihood-histograms.pdf}{
    Histograms of the potential scale reduction factor (the square root of
    Equation 1.1 in Brooks and Gelman \citeyear{Brooks1998}) for the log
    likelihood across the three MCMC chains run for each simulated \dataset.
    \validationfiguregriddescription
    The mean and range across the 500 \datasets is indicated
    in the upper right corner of each plot.
    \weusedmatplotlib
}{fig:valpsrflikelihood}

\siFigure{../../../simulations/validation/plots/psrf-div-time-histograms.pdf}{
    Histograms of the potential scale reduction factor (the square root of
    Equation 1.1 in Brooks and Gelman \citeyear{Brooks1998}) for the divergence
    times across the three MCMC chains run for each simulated \dataset.
    \validationfiguregriddescription
    The mean and range across the 500 \datasets is indicated
    in the upper right corner of each plot.
    \weusedmatplotlib
}{fig:valpsrfdivtimes}

\siFigure{../../../simulations/validation/plots/ess-ln-likelihood-histograms.pdf}{
    Histogram of the estimated effective sample sizes \citep{Gong2014} for the
    log likelihood across the three MCMC chains run for each simulated
    \dataset.
    \validationfiguregriddescription
    The mean and range across the 500 \datasets is indicated in the upper right
    corner of each plot.
    \weusedmatplotlib
}{fig:valesslikelihood}

\siFigure{../../../simulations/validation/plots/ess-div-time-histograms.pdf}{
    Histogram of the estimated effective sample sizes \citep{Gong2014} for the
    divergence times across the three MCMC chains run for each simulated
    \dataset.
    \validationfiguregriddescription
    The mean and range across the 500 \datasets is indicated in the upper right
    corner of each plot.
    \weusedmatplotlib
}{fig:valessdivtimes}

\siFigure{../../../simulations/validation/plots/div-time-ess-vs-error-scatter.pdf}{
    Plots of the estimated effective sample sizes \citep{Gong2014} for the
    divergence times against absolute error of the divergence-time estimates.
    \validationfiguregriddescription
    \weusedmatplotlib
}{fig:valessvserror}

\siFigure{../../../simulations/validation/plots/linkage-100k-div-time-scatter.pdf}{
    Assessing the \jroedit{a}{e}ffect of linked sites on the
    the accuracy and precision of divergence time estimates (in units of
    expected subsitutions per site).
    The columns, from left to right, show the results when loci are simulated
    with 100, 500, and 1000 linked sites.
    For each simulated \dataset, each of three population pairs has
    100,000 sites total.
    The rows show the results when (top) all sites, (middle) all variable
    sites, and (bottom) at most one variable site per locus are analyzed.
    \accuracyscatterplotannotations{\comparisondivtime{}}
    \weusedmatplotlib
}{fig:linkagedivtimes100k}

\siFigure{../../../simulations/validation/plots/linkage-100k-leaf-pop-size-scatter.pdf}{
    Assessing the \jroedit{a}{e}ffect of linked sites on the accuracy and precision of
    estimates of descendant (leaf) population sizes (scaled by the mutation
    rate).
    The columns, from left to right, show the results when loci are simulated
    with 100, 500, and 1000 linked sites.
    For each simulated \dataset, each of three population pairs has
    100,000 sites total.
    The rows show the results when (top) all sites, (middle) all variable
    sites, and (bottom) at most one variable site per locus are analyzed.
    \accuracyscatterplotannotations{\epopsize{}\murate{}}
    \weusedmatplotlib
}{fig:linkageleafsizes100k}

\siFigure{../../../simulations/validation/plots/linkage-500k-leaf-pop-size-scatter.pdf}{
    Assessing the \jroedit{a}{e}ffect of linked sites on the accuracy and precision of
    estimates of descendant (leaf) population sizes (scaled by the mutation
    rate).
    The columns, from left to right, show the results when loci are simulated
    with 100, 500, and 1000 linked sites.
    For each simulated \dataset, each of three population pairs has
    500,000 sites total.
    The rows show the results when (top) all sites, (middle) all variable
    sites, and (bottom) at most one variable site per locus are analyzed.
    \accuracyscatterplotannotations{\epopsize{}\murate{}}
    \weusedmatplotlib
}{fig:linkageleafsizes500k}

\siFigure{../../../simulations/validation/plots/linkage-100k-root-pop-size-scatter.pdf}{
    Assessing the \jroedit{a}{e}ffect of linked sites on the accuracy and precision of
    estimates of ancestral (root) population size (scaled by the mutation
    rate).
    The columns, from left to right, show the results when loci are simulated
    with 100, 500, and 1000 linked sites.
    For each simulated \dataset, each of three population pairs has
    100,000 sites total.
    The rows show the results when (top) all sites, (middle) all variable
    sites, and (bottom) at most one variable site per locus are analyzed.
    \accuracyscatterplotannotations{\epopsize{}\murate{}}
    \weusedmatplotlib
}{fig:linkagerootsizes100k}

\siFigure{../../../simulations/validation/plots/linkage-500k-root-pop-size-scatter.pdf}{
    Assessing the \jroedit{a}{e}ffect of linked sites on the accuracy and precision of
    estimates of ancestral (root) population size (scaled by the mutation
    rate).
    The columns, from left to right, show the results when loci are simulated
    with 100, 500, and 1000 linked sites.
    For each simulated \dataset, each of three population pairs has
    500,000 sites total.
    The rows show the results when (top) all sites, (middle) all variable
    sites, and (bottom) at most one variable site per locus are analyzed.
    \accuracyscatterplotannotations{\epopsize{}\murate{}}
    \weusedmatplotlib
}{fig:linkagerootsizes500k}

\siFigure{\ifuserasterizedplotsinsi{../../../simulations/validation/plots/linkage-500k-models-compressed.pdf}{../../../simulations/validation/plots/linkage-500k-models.pdf}}{
    \jroedit{}{Assessing the effect of linked sites on estimating the
    divergence model.
    The columns, from left to right, show the results when loci are simulated
    with 100, 500, and 1000 linked sites.
    For each simulated \dataset, each of three population pairs has
    500,000 sites total.
    The rows show the results when (top) all sites, (middle) all variable
    sites, and (bottom) at most one variable site per locus are analyzed.
    \modelshadingdescription
    \modelplotannotations
    \weusedmatplotlib
    }
}{fig:linkagemodels500k}

\siFigure{\ifuserasterizedplotsinsi{../../../simulations/validation/plots/linkage-100k-nevents-compressed.pdf}{../../../simulations/validation/plots/linkage-100k-nevents.pdf}}{
    Assessing the \jroedit{a}{e}ffect of linked sites on estimating the number of divergence
    events.
    The columns, from left to right, show the results when loci are simulated
    with 100, 500, and 1000 linked sites.
    For each simulated \dataset, each of three population pairs has
    100,000 sites total.
    The rows show the results when (top) all sites, (middle) all variable
    sites, and (bottom) at most one variable site per locus are analyzed.
    \neventsshadingdescription
    \neventplotannotations
    \weusedmatplotlib
}{fig:linkagenevents100k}

\siFigure{\ifuserasterizedplotsinsi{../../../simulations/validation/plots/linkage-100k-models-compressed.pdf}{../../../simulations/validation/plots/linkage-100k-models.pdf}}{
    \jroedit{}{Assessing the effect of linked sites on estimating the
    divergence model.
    The columns, from left to right, show the results when loci are simulated
    with 100, 500, and 1000 linked sites.
    For each simulated \dataset, each of three population pairs has
    100,000 sites total.
    The rows show the results when (top) all sites, (middle) all variable
    sites, and (bottom) at most one variable site per locus are analyzed.
    \modelshadingdescription
    \modelplotannotations
    \weusedmatplotlib
    }
}{fig:linkagemodels100k}


\siFigure{../../../simulations/validation/plots/linkage-100k-number-of-variable-sites-histograms.pdf}{
    The number of variable characters in the simulated \datasets with 100,000
    characters per pair of populations that were linked in loci of length (left
    column) 100, (middle column) 500, and (right column) 1000 sites.
    The first row shows all the variable sites, whereas the second row shows
    when at most one variable site per locus is randomly choosen.
    The mean and range across the 500 \datasets is indicated
    in the upper right corner of each plot.
    \weusedmatplotlib
}{fig:linkagenumberofvariablesites100k}

\siFigure{../../../simulations/validation/plots/linkage-500k-number-of-variable-sites-histograms.pdf}{
    The number of variable characters in the simulated \datasets with 500,000
    characters per pair of populations that were linked in loci of length (left
    column) 100, (middle column) 500, and (right column) 1000 sites.
    The first row shows all the variable sites, whereas the second row shows
    when at most one variable site per locus is randomly choosen.
    The mean and range across the 500 \datasets is indicated
    in the upper right corner of each plot.
    \weusedmatplotlib
}{fig:linkagenumberofvariablesites500k}


\siFigure{../../../simulations/validation/plots/missing-data-leaf-pop-size-scatter.pdf}{
    Assessing the \jroedit{a}{e}ffect of missing data on the the accuracy and precision of
    estimates of descendant (leaf) population sizes (scaled by the mutation
    rate).
    \missingdatafigurecolumndescription
    \comparisoncolumndescription{S}{fig:valleafsizes}
    \missingdatafigurerowdescription
    \accuracyscatterplotannotations{\epopsize{}\murate{}}
    All simulated \datasets had three pairs of populations.
    \weusedmatplotlib
}{fig:missingleafsizes}

\siFigure{../../../simulations/validation/plots/missing-data-root-pop-size-scatter.pdf}{
    Assessing the \jroedit{a}{e}ffect of missing data on the the accuracy and precision of
    estimating ancestral (root) population size (scaled by the mutation rate).
    \missingdatafigurecolumndescription
    \comparisoncolumndescription{S}{fig:valrootsizes}
    \missingdatafigurerowdescription
    \accuracyscatterplotannotations{\epopsize{}\murate{}}
    All simulated \datasets had three pairs of populations.
    \weusedmatplotlib
}{fig:missingrootsizes}

\siFigure{\ifuserasterizedplotsinsi{../../../simulations/validation/plots/missing-data-models-compressed.pdf}{../../../simulations/validation/plots/missing-data-models.pdf}}{
    \jroedit{}{Assessing the effect of missing data on the ability of the new
    method to estimate the divergence model.
    \missingdatafigurecolumndescription
    \comparisoncolumndescription{}{fig:valmodels}
    \missingdatafigurerowdescription
    \modelshadingdescription
    \modelplotannotations
    All simulated \datasets had three pairs of populations.
    \weusedmatplotlib
    }
}{fig:missingmodels}


\siFigure{../../../simulations/validation/plots/filtered-data-leaf-pop-size-scatter.pdf}{
    Assessing the \jroedit{a}{e}ffect of an acquisition bias against rare allele patterns on
    the accuracy and precision of estimating descendant (leaf) population sizes
    (scaled by the mutation rate).
    \filtereddatafigurecolumndescription
    \comparisoncolumndescription{S}{fig:valleafsizes}
    \filtereddatafigurerowdescription
    \accuracyscatterplotannotations{\epopsize{}\murate{}}
    All simulated \datasets had three pairs of populations.
    \weusedmatplotlib
}{fig:filteredleafsizes}

\siFigure{../../../simulations/validation/plots/filtered-data-root-pop-size-scatter.pdf}{
    Assessing the \jroedit{a}{e}ffect of an acquisition bias against rare allele patterns on
    the accuracy and precision of estimating ancestral (root) population size
    (scaled by the mutation rate).
    \filtereddatafigurecolumndescription
    \comparisoncolumndescription{S}{fig:valrootsizes}
    \filtereddatafigurerowdescription
    \accuracyscatterplotannotations{\epopsize{}\murate{}}
    All simulated \datasets had three pairs of populations.
    \weusedmatplotlib
}{fig:filteredrootsizes}

\siFigure{\ifuserasterizedplotsinsi{../../../simulations/validation/plots/filtered-data-models-compressed.pdf}{../../../simulations/validation/plots/filtered-data-models.pdf}}{
    \jroedit{}{Assessing the effect of an acquisition bias against
    rare allele patterns on the ability of the new method to estimate the
    divergence model.
    \filtereddatafigurecolumndescription
    \comparisoncolumndescription{}{fig:valmodels}
    \filtereddatafigurerowdescription
    \modelshadingdescription
    \modelplotannotations
    All simulated \datasets had three pairs of populations.
    \weusedmatplotlib
    }
}{fig:filteredmodels}


\siFigure{../../../bake-off/plots/leaf-population-size-scatter.pdf}{
    Comparing the accuracy and precision of estimates of the descendant (leaf)
    population sizes between (left) the new full-likelihood Bayesian method,
    \ecoevolity, and (right) the approximate-likelihood Bayesian method,
    \dppmsbayes.
    Each plotted circle and associated error bars represent the posterior mean
    and 95\% credible interval.
    Each plot consists of 3000 estimates---500 simulated \datasets, each with
    three pairs of populations.
    The simulated character matrix for each population pair consisted of 200 loci,
    each with 200 linked sites (40,000 characters total).
    \accuracyscatterplotannotations{\epopsize{}\murate{}}
    \weusedmatplotlib
}{fig:bakeoffleafsizes}

\siFigure{../../../bake-off/plots/root-population-size-scatter.pdf}{
    Comparing the accuracy and precision of estimates of the ancestral (root)
    population size between (left) the new full-likelihood Bayesian method,
    \ecoevolity, and (right) the approximate-likelihood Bayesian method,
    \dppmsbayes.
    Each plotted circle and associated error bars represent the posterior mean
    and 95\% credible interval.
    Each plot consists of 1500 estimates---500 simulated \datasets, each with
    three pairs of populations.
    The simulated character matrix for each population pair consisted of 200 loci,
    each with 200 linked sites (40,000 characters total).
    \accuracyscatterplotannotations{\epopsize{}\murate{}}
    \weusedmatplotlib
}{fig:bakeoffrootsizes}

\siFigure{\ifuserasterizedplotsinsi{../../../bake-off/plots/model-compressed.pdf}{../../../bake-off/plots/model.pdf}}{
    Comparing the ability to estimate the divergence model between
    (left) the new full-likelihood Bayesian method, \ecoevolity, and (right)
    the approximate-likelihood Bayesian method, \dppmsbayes.
    Each plot shows the results of the analyses of 500 simulated \datasets;
    the number of \datasets that fall within each possible cell
    of true versus estimated model is shown, and cells with
    more \datasets are shaded darker.
    Each simulated \dataset contained three pairs of populations, and the
    simulated character matrix for each pair consisted of 200 loci, each with
    200 linked sites (40,000 characters total).
    \modelplotannotations
    \weusedmatplotlib
}{fig:bakeoffmodels}


\siFigure{../../../gekko-results/sumevents-nopoly.pdf}{
    The prior (light bars) and approximated posterior (dark bars) probabilities
    of the number of divergence events across \spp{Gekko} pairs of populations,
    under eight different combinations of prior on the divergence times (rows)
    and the concentration parameter of the Dirichlet process (columns).
    For these analyses, constant characters were included, but all characters
    with more than two alleles were removed.
    The Bayes factor for each number of divergence times is given above the
    corresponding bars.
    Each Bayes factor compares the corresponding number of events 
    to all other possible numbers of divergence events.
    \weusedggplot
}{fig:gekkonopolynevents}

\siFigure{../../../gekko-results/sumsizes.pdf}{
    The approximate marginal posterior densities of population sizes for each
    \spp{Gekko} pair of populations,
    under eight different combinations of prior on the divergence times (rows)
    and the concentration parameter of the Dirichlet process (columns).
    For these analyses, constant characters were included, but all characters
    with more than two alleles were recoded as biallelic.
    \weusedggridges
}{fig:gekkosizes}

\siFigure{../../../gekko-results/sumtimes-nopoly.pdf}{
    The approximate marginal posterior densities of divergence times for each
    \spp{Gekko} pair of populations,
    under eight different combinations of prior on the divergence times (rows)
    and the concentration parameter of the Dirichlet process (columns).
    For these analyses, constant characters were included, but all characters
    with more than two alleles were removed.
    \weusedggridges
}{fig:gekkonopolydivtimes}

\siFigure{../../../gekko-results/sumsizes-nopoly.pdf}{
    The approximate marginal posterior densities of population sizes for each
    \spp{Gekko} pair of populations,
    under eight different combinations of prior on the divergence times (rows)
    and the concentration parameter of the Dirichlet process (columns).
    For these analyses, constant characters were included, but all characters
    with more than two alleles were removed.
    \weusedggridges
}{fig:gekkonopolysizes}

\siFigure{../../../gekko-results/sumtimes-varonly.pdf}{
    The approximate marginal posterior densities of divergence times for each
    \spp{Gekko} pair of populations,
    under eight different combinations of prior on the divergence times (rows)
    and the concentration parameter of the Dirichlet process (columns).
    For these analyses, constant characters were excluded, and all characters
    with more than two alleles were recoded as biallelic.
    \weusedggridges
}{fig:gekkovaronlydivtimes}

\siFigure{../../../gekko-results/sumsizes-varonly.pdf}{
    The approximate marginal posterior densities of population sizes for each
    \spp{Gekko} pair of populations,
    under eight different combinations of prior on the divergence times (rows)
    and the concentration parameter of the Dirichlet process (columns).
    For these analyses, constant characters were excluded, and all characters
    with more than two alleles were recoded as biallelic.
    \weusedggridges
}{fig:gekkovaronlysizes}

\siFigure{../../../gekko-results/sumevents-varonly.pdf}{
    The prior (light bars) and approximated posterior (dark bars) probabilities
    of the number of divergence events across \spp{Gekko} pairs of populations,
    under eight different combinations of prior on the divergence times (rows)
    and the concentration parameter of the Dirichlet process (columns).
    For these analyses, constant characters were excluded, and all characters
    with more than two alleles were recoded as biallelic.
    The Bayes factor for each number of divergence times is given above the
    corresponding bars.
    Each Bayes factor compares the corresponding number of events 
    to all other possible numbers of divergence events.
    \weusedggplot
}{fig:gekkovaronlynevents}

\siFigure{../../../gekko-results/sumtimes-nopoly-varonly.pdf}{
    The approximate marginal posterior densities of divergence times for each
    \spp{Gekko} pair of populations,
    under eight different combinations of prior on the divergence times (rows)
    and the concentration parameter of the Dirichlet process (columns).
    For these analyses, constant characters and all characters
    with more than two alleles were removed.
    \weusedggridges
}{fig:gekkonopolyvaronlydivtimes}

\siFigure{../../../gekko-results/sumsizes-nopoly-varonly.pdf}{
    The approximate marginal posterior densities of population sizes for each
    \spp{Gekko} pair of populations,
    under eight different combinations of prior on the divergence times (rows)
    and the concentration parameter of the Dirichlet process (columns).
    For these analyses, constant characters and all characters
    with more than two alleles were removed.
    \weusedggridges
}{fig:gekkonopolyvaronlysizes}

\siFigure{../../../gekko-results/sumevents-nopoly-varonly.pdf}{
    The prior (light bars) and approximated posterior (dark bars) probabilities
    of the number of divergence events across \spp{Gekko} pairs of populations,
    under eight different combinations of prior on the divergence times (rows)
    and the concentration parameter of the Dirichlet process (columns).
    For these analyses, constant characters and all characters
    with more than two alleles were removed.
    The Bayes factor for each number of divergence times is given above the
    corresponding bars.
    Each Bayes factor compares the corresponding number of events 
    to all other possible numbers of divergence events.
    \weusedggplot
}{fig:gekkonopolyvaronlynevents}
