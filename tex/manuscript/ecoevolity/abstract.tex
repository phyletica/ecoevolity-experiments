A challenge to understanding biological diversification is accounting for
large-scale processes that cause multiple, co-distributed lineages to
co-speciate.
Such processes predict non-independent, temporally clustered divergences across
taxa.
Approximate-likelihood Bayesian computation (ABC) approaches to inferring such
patterns from comparative genetic data are very sensitive to prior assumptions
and often biased toward estimating shared divergences.
We introduce a full-likelihood Bayesian approach, \ecoevolity, which takes full
advantage of information in genomic data.
By analytically integrating over gene trees, we are able to directly calculate
the likelihood of the population history from genomic data, and efficiently
sample the model-averaged posterior via Markov chain Monte Carlo algorithms.
Using simulations, we find that the new method is much more accurate and
precise at estimating the number and timing of divergence events across pairs
of populations than existing approximate-likelihood methods.
Our full Bayesian approach also requires several orders of magnitude less
computational time than existing ABC approaches.
We find that despite assuming unlinked characters (e.g., unlinked
single-nucleotide polymorphisms), the new method performs better if this
assumption is violated in order to retain the constant characters of whole
linked loci.
In fact, retaining constant characters allows the new method to robustly
estimate the correct number of divergence events with high posterior
probability in the face of character-acquisition biases, which commonly
plague loci assembled from reduced-representation genomic libraries.
We apply our method to genomic data from four pairs of insular populations of
\spp{Gekko} lizards from the Philippines that are \emph{not} expected to have
co-diverged.
Despite all four pairs diverging very recently, our method strongly supports
that they diverged independently, and these results are robust to very
disparate prior assumptions.
