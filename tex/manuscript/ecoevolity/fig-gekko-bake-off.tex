\embedWidthFigure{1.0}{../../../gekko-bake-off/results/comparison.pdf}{
    The results of the (A \& B) new full-likelihood Bayesian method,
    \ecoevolity, and (C \& D) approximate-likelihood Bayesian method,
    \dppmsbayes, when applied to 200 RADseq loci randomly sampled from three of
    the pairs of \spp{Gekko} populations.
    Plots B \& D show the estimated marginal posterior densities of divergence
    times for each pair of \spp{Gekko} populations.
    Plots A \& C show the approximated prior (light bars) and posterior (dark
    bars) probabilities of the number of divergence events across the pairs of
    \spp{Gekko} populations.
    The Bayes factor for each number of divergence times is given above the
    corresponding bars.
    Each Bayes factor compares the corresponding number of events to all other
    possible numbers of divergence events.
    \weusedggridges
}{fig:gekkobakeoff}
