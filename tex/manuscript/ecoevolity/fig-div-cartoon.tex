\embedWidthFigure{1.0}{../../images/div-cartoon/div-cartoon.pdf}{
    A cartoon depiction of the inference problem for three pairs of
    insular lizard populations.
    Three ancestral species of lizards co-occured on a paleo-island that was
    fragmented into two islands by a rise in sea levels at \divtime[1].
    The island fragmentation caused the second and third (from the top)
    lineages to co-diverge; the first lineage diverged later (at \divtime[2])
    via over-water dispersal.
    The five possible divergence models are shown to the right, with the
    correct model indicated.
    The divergence-time parameters (\divtime[1] and \divtime[2])
    and the pair-specific divergence times (\comparisondivtime[1],
    \comparisondivtime[2], and \comparisondivtime[3])
    are shown.
    The third population pair shows the notation used in the text for the
    biallelic character data
    ($\leafallelecounts, \leafredallelecounts = (\allelecount[1],
    \redallelecount[1]), (\allelecount[2], \redallelecount[2])$)
    and effective sizes of the ancestral
    (\epopsize[\rootpopindex{}]) and descendant
    (\epopsize[\descendantpopindex{1}] and \epopsize[\descendantpopindex{2}])
    populations.
    The lizard silhouette for the middle pair is from pixabay.com, and the
    other two are from phylopic.org; all were licensed under the Creative
    Commons (CC0) 1.0 Universal Public Domain Dedication.
}{fig:divCartoon}
