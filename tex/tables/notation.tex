\begin{table}[htbp]
    \sffamily
    \small
    % \addtolength{\tabcolsep}{-0.1cm}
\caption{
    Summary of the notation.
}
    \centering
    \begin{tabular}{ l >{\raggedright\hangindent=0.5cm}m{14cm} }
    % \begin{tabular}{ l l }
        \toprule
        \textbf{Symbol} & \textbf{Description} \tn
        \midrule
        \ncomparisons{} & The number of population pairs being compared.
        \tn
        \nevents{} & The number of divergence events.
        \tn
        \allelecount, \redallelecount & The number of copies of a locus sampled
            from a population, and the number of those copies that are the ``red''
            allele.
            \tn
        \leafallelecounts, \leafredallelecounts & The allele counts from both
            populations of a pair (i.e., $\leafallelecounts, \leafredallelecounts =
            (\allelecount[1], \redallelecount[1]), 
            (\allelecount[2], \redallelecount[2])$).
            \tn
        \comparisondata[i] & The allele counts across all the loci from a population pair $i$.
        \tn
        \genetree & A gene tree with branch lengths.
        \tn
        \murate & The rate of mutation.
        \tn
        \rgmurate & Relative rate of mutating from the ``red'' to ``green'' state.
        \tn
        \grmurate & Relative rate of mutating from the ``green'' to ``red'' state.
        \tn
        \gfreq & The stationary frequency of the ``green'' state.
        \tn
        \comparisondivtime[i] & The time in the past when two populations of pair $i$ diverged.
        \tn
        \bottomrule
    \end{tabular}
    \label{table:notation}
\end{table}

