\begin{table}[htbp]
    \sffamily
    \small
    \rowcolors{2}{}{mygray}
    \addtolength{\tabcolsep}{-0.1cm}
\caption{
    A key to some of the notation used in the text.
}
    \centering
    \begin{tabular}{ l >{\raggedright\hangindent=0.5cm}m{14cm} }
        \toprule
        \textbf{Symbol} & \textbf{Description} \tn
        \midrule
        \ncomparisons{} & The number of population pairs being compared.
        \tn
        \nevents{} & The number of divergence times (or ``events'') across the
            population pairs being compared.
        \tn
        \comparisondivtime[i] & The time in the past when the two populations
            of pair $i$ diverged.
        \tn
        \divtime & The time of a divergence event at which one or more pairs of
            populations diverged.
        \tn
        \divtimemodel & The divergence model, which comprises the divergence
            times and the mapping of the population pairs to those times.
        \tn
        \divtimes & All of the unique divergence times in the model
            ($\divtimes = \divtime[1], \ldots, \divtime[\nevents]$).
        \tn
        \divtimesets & The mapping of population pairs to divergence events,
            but not the times of the events.
        \tn
        \basedistribution & The base distribution of the Dirichlet process.
        \tn
        \concentration & The concentration parameter of the Dirichlet process.
        \tn
        \allelecount, \redallelecount & The number of copies of a locus sampled
            from a population, and the number of those copies that are the ``red''
            allele.
            \tn
        \leafallelecounts, \leafredallelecounts & The allele counts from both
            populations of a pair (i.e., $\leafallelecounts, \leafredallelecounts =
            (\allelecount[1], \redallelecount[1]), 
            (\allelecount[2], \redallelecount[2])$).
            \tn
        \comparisondata[i] & The allele counts across all the loci from
            population pair $i$. I.e., all of the characters being analyzed for
            population pair $i$.
            \tn
        \nloci & The number of loci collected for a pair of populations.
        \tn
        \alldata & All of the data being analyzed, i.e., the
            character matrices from all population pairs.
        \tn
        \genetree & A gene tree with branch lengths.
        \tn
        \murate & The rate of mutation.
        \tn
        \rgmurate & Relative rate of mutating from the ``red'' to ``green'' state.
        \tn
        \grmurate & Relative rate of mutating from the ``green'' to ``red'' state.
        \tn
        \gfreq & The stationary frequency of the ``green'' state.
        \tn
        \epopsize[\rootpopindex] & The effective size of the ancestral population.
        \tn
        \epopsize[\descendantpopindex{1}], \epopsize[\descendantpopindex{2}] &
            The effective sizes of the two descendant populations of a pair.
        \tn
        \comparisonpopsizes & Shorthand notation for all three effective
            population sizes for a pair (ancestral and the two descendant
            populations).
        \tn
        \sptree & The species tree for a pair of populations. This comprises the
            three effective population sizes (ancestral and the two descendant)
            and the time of divergence.
            \tn
        \bottomrule
    \end{tabular}
    \label{table:notation}
\end{table}

