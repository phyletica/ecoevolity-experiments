% \begin{landscape}
\begin{table}[htbp]
\sffamily
% \small
\addtolength{\tabcolsep}{-0.5mm}
\footnotesize
\caption{
    A summary of the data collected from the pairs of \spp{Cyrtodactylus} and
    \spp{Gekko} populations from the Philippines.
    Each row represents a pair of populations sampled from two islands that
    were never connected during low sea levels of glacial periods.
}
\centering
\begin{tabular}{ @{}l l l l l l l l l@{} }
Species
        & Island 1
        & Island 2
        & \multicolumn{2}{l}{Sample sizes}
        & \# loci
        & \# sites
        & \# variable
        & \# polyallelic
        \\
\hline
\spp{G.\ crombota-rossi}
        & Babuyan Claro
        & Calayan
        & 5
        & 5
        & 16,901
        & 1,538,408
        & 5737
        & 50
        \\
\spp{G.\ mindorensis}
        & Lubang
        & Luzon
        & 5
        & 4
        & 18,137
        & 1,651,186
        & 12,092
        & 68
        \\
\spp{G.\ mindorensis}
        & Maestre De Campo
        & Masbate
        & 3
        & 3
        & 15,993
        & 1,455,238
        & 11,845
        & 27
        \\
\spp{G.\ sp.\ B-sp.\ A}
        & Camiguin Norte
        & Dalupiri
        & 5
        & 5
        & 15,199
        & 1,383,596
        & 5612
        & 31
        \\
\hline
\end{tabular}
\label{table:gekkocomparisons}
\end{table}
% \end{landscape}
