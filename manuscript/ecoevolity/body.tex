\section{Introduction}

One of the primary goals of phylogeography is to understand to what degree the
landscape, and changes to it, explain biological diversity within and among
species, and across communities.
A major component of this is understanding whether landscape-level processes
have driven diversification across multiple, independent evolutionary lineages.
Such processes are expected to generate patterns of divergence times
that are difficult to explain by lineage-specific processes of 
diversification.
Specifically, finding that divergence times are temporally clustered across
independent evolutionary lineages provides compelling evidence that a shared,
landscape-level process was responsible for the lineages diverging.
For example, the fragmentation of an environment, like an island, forest, or
watershed, can cause many lineages to co-diverge over a very short interval
relative to evolutionary timescales, such that a model that treats them as
simultaneous will be a better explanation of the genetic variation than a model
that treats the divergence of each lineage as independent \figureNeeded.
As a result, we stand to learn a lot about processes of diversification from a
robust method for inferring the temporal pattern of divergences across
multiple, independent lineages.
\thought{It would be nice to reference a cartoon example above}.

\begin{linenomath}
Fundamentally, this is a problem of model choice:
How many unique divergences best explain the genetic variation within and
between the populations of each lineage?
If we nave \ncomparisons pairs of populations, we would like to assign them to an
unknown number of divergence times, \nevents, which can range from one to
\ncomparisons.
For a given number of divergence times, the Stirling number of the second kind
tells us the number of ways of assigning the taxa to the divergence times
(i.e., the number of models with \nevents divergence-time parameters):
\begin{equation}
    S_2(\ncomparisons, \nevents) = 
    \frac{1}{\nevents!} \sum_{i = 0}^{\nevents - 1} (-1)^{i}
    \binom{\nevents}{i} (\nevents - i)^{\ncomparisons}.
    \label{eq:stirling2}
\end{equation}
When the number of divergence times is unknown, to get the total
number of possible divergence models, we need to sum over
all possible values of \nevents, which is Bell's number \citep{Bell1934}:
\begin{equation}
    B_{\ncomparisons} = \sum_{\nevents = 1}^{\ncomparisons}
    S_2(\ncomparisons, \nevents).
    \label{eq:bell}
\end{equation}
\end{linenomath}

\section{Methods}

\subsection{The data}
For each pair of populations that we wish to compare, we assume that we have
collected orthologous genetic markers with at most two states.
We will refer to these as `biallelic characters', but note that this can
include constant characters.
We assume each character is effectivley unlinked, i.e., each marker evolved
along a gene tree that is independent of the others, conditional on the
population history.
Examples include well-spaced single nucleotide polymorhphisms (SNPs) or
amplified fragment length polymorphisms (AFLPs).

So, for each population and for each marker we sample \allelecount
copies of the locus, \redallelecount of which are copies of the ``red''
allele and the remaining $\allelecount - \redallelecount$ are
copies of the ``green'' allele;
\redallelecount can range from zero to \allelecount.
Thus, for each population of a pair for a given marker we have a count of the
total sampled gene copies and how many of those are the ``red'' allele.

We will use \leafallelecounts and \leafredallelecounts to denote allele counts
from both populations of a pair; i.e., 
$\leafallelecounts, \leafredallelecounts = (\allelecount[1], \redallelecount[1]), 
(\allelecount[2], \redallelecount[2])$.
We will use \comparisondata[i] to denote these counts for all the loci from
population pair $i$.
% and \alldata to represent the data across all the pairs of
% populations of which we wish to compare the divergence times.



\subsection{The model}

\subsubsection{The evolution of markers}

We begin unpacking our model by first focusing on a single pair of populations.

We assume a finite-sites, continuous-time Markov chain (CTMC) model for the
evolution of the biallelic characters along a gene tree with branch lengths
\genetree.
As the marker evolves along the gene tree, forward in time, there is an
relative instantaneous rate of \rgmurate mutating from the red state to the
green state, and a corresponding relative rate \grmurate of mutation from green
to read.
The stationary frequency of the red and green state is then
$\grmurate / (\rgmurate + \grmurate)$
and
$\rgmurate / (\rgmurate + \grmurate)$, respectively.
Thus, if given the frequency of the green allele, \gfreq, we can obtain the
relative rates of mutation between the two states.
We will denote the overall rate of mutation as \murate.
If $\murate = 1$, the gene tree branch lengths, and time in general, is in
units of the expected mutations per site.
If a mutation rate per site per generation is given, time is in units of
generations.
For a given pair of populations, the \murate is redundant with the branch
lengths of the gene tree.
However, we introduce the notation here, because it will be useful later if we
want to allow rate variation among pairs of populations.
% For now, we will assume $\murate = 1$ so that the gene tree branch lengths, and
% time in general, is in units of the expected mutations per site.

\subsubsection{The evolution of gene trees}

We assume that each marker sampled from a pair of populations evolved within a
simple ``species'' tree with one ancestral root population that diverged into
two descendant (terminal) branches at time \comparisondivtime.
If the $\murate$ is given, time is in units of generations or years, however,
if $\murate$ is set to one, time is in units of expected mutations per site.
We will use
$\comparisonpopsizes = \epopsize[\rootpopindex],
\epopsize[\descendantpopindex{1}], \epopsize[\descendantpopindex{2}]$
to denote
the effective sizes of the root and two descendant populations, respectively.
We will also use $\sptree = \comparisonpopsizes, \comparisondivtime$ as shorthand for the
species tree.

\subsubsection{The likelihood}

Given \murate, \gfreq, \comparisondivtime and \comparisonpopsizes, the
likelihood of the observed data at a marker (\allelecount and \redallelecount),
is the probability of the site pattern given a gene tree multiplied by the
probability of the gene given the species tree, summed over all possible gene
tree topologies and integrated over all possbile gene tree branch lengths
\begin{equation}
    \pr(\leafallelecounts, \leafredallelecounts \given \sptree, \murate, \gfreq)
    =
    \int_{\genetree}
    \pr(\leafallelecounts, \leafredallelecounts \given \genetree, \murate, \gfreq)
    \pr(\genetree, \murate, \gfreq \given \sptree)
    \diff{\genetree}.
    \label{eq:markerlikelihood}
\end{equation}
We take advantage of the elegant mathematical work of \citep{Bryant2012} to
calculate the integral over gene trees analytically, allowing us to compute the
likelihood of the species tree directly from the site patterns under a
coalescent model.
\thought{For completeness, should I lay out the math and algorithm for
    calculating the likelihood in the supplemental materials?}

Assuming independence among loci, we can calculate the probability of the
\nloci loci given the specie tree by simply taking the product over them
\begin{equation}
    \pr(\comparisondata \given \sptree, \murate, \gfreq)
    =
    \prod_{i=1}^{\nloci}
    \pr(\leafallelecounts[i], \leafredallelecounts[i] \given \sptree, \murate, \gfreq).
    \label{eq:comparisonlikelihood}
\end{equation}
Finally, the likelihood across all of our pairs is simply the product of the
likelihood of each pair,
\begin{equation}
    \pr(
    \alldata
    \given
    \sptrees,
    \murates,
    \gfreqs)
    =
    \prod_{i=1}^{\ncomparisons}
    \pr(\comparisondata[i] \given \sptree[i], \murate[i], \gfreq[i]),
    \label{eq:collectionlikelihood}
\end{equation}
where
$\alldata = \comparisondata[1], \comparisondata[2], \ldots, \comparisondata[\ncomparisons]$,
$\sptrees = \sptree[1], \sptree[2], \ldots, \sptree[\ncomparisons]$,
$\murates = \murate[1], \murate[2], \ldots, \murate[\ncomparisons]$,
and
$\gfreqs = \gfreq[1], \gfreq[2], \ldots, \gfreq[\ncomparisons]$.


\subsubsection{Correcting for missing constant characters}
If we only have variable characters, we need to correct the likelihood for the
fact that we are not sampling any constant characters.

\begin{equation}
\begin{split}
    \pr(\leafallelecounts, \leafredallelecounts \given \sptree, \murate, \gfreq, \textrm{variable})
    & =
    \frac{
        \pr(\leafallelecounts, \leafredallelecounts \given \sptree, \murate, \gfreq)
    }{
        \pr(\textrm{variable} \given \sptree, \murate, \gfreq)
    } \\
    & =
    \frac{
        \pr(\leafallelecounts, \leafredallelecounts \given \sptree, \murate, \gfreq)
    }{
        1 - \pr(\textrm{constant} \given \sptree, \murate, \gfreq)
    } \\
    & =
    \frac{
        \pr(\leafallelecounts, \leafredallelecounts \given \sptree, \murate, \gfreq)
    }{
        1 - \pr(\leafallelecounts \textrm{ all red} \given \sptree, \murate, \gfreq)
        - \pr(\leafallelecounts \textrm{ all green} \given \sptree, \murate, \gfreq)
    }
    \label{eq:variablemarkerlikelihood}
\end{split}
\end{equation}

\begin{equation}
    \pr(\comparisondata \given \sptree, \murate, \gfreq, \textrm{variable})
    =
    \prod_{i=1}^{\nloci}
    \frac{
        \pr(\leafallelecounts[i], \leafredallelecounts[i] \given \sptree, \murate, \gfreq)
    }{
        1 - \pr(\leafallelecounts[i] \textrm{ all red} \given \sptree, \murate, \gfreq)
        - \pr(\leafallelecounts[i] \textrm{ all green} \given \sptree, \murate, \gfreq)
    }.
    \label{eq:variablecomparisonlikelihood}
\end{equation}
This is a bit different than the correction done in SNAPP.
If we use \maxleafallelecounts to denote the maximum number
of gene copies sampled from each population, then SNAPP does
\begin{equation}
    \pr(\comparisondata \given \sptree, \murate, \gfreq, \textrm{variable})
    =
    \frac{
        \prod_{i=1}^{\nloci}
        \pr(\leafallelecounts[i], \leafredallelecounts[i] \given \sptree, \murate, \gfreq).
    }{
        (1 - \pr(\maxleafallelecounts \textrm{ all red} \given \sptree, \murate, \gfreq)
        - \pr(\maxleafallelecounts \textrm{ all green} \given \sptree, \murate, \gfreq))^{\nloci}
    }.
    \label{eq:snappvariablecomparisonlikelihood}
\end{equation}
These are equivalent if the same number of samples are collected across all
loci for each population (i.e., no missing data), but will deviate if fewer
gene copies are sampled for atleast one locus.
Thus, identical likelihoods between SNAPP and our method should not be expected
when analyzing variable-only data.

\subsection{Bayesian inference}

We can obtain the posterior probability distribution by naively plugging the
likelihood in Equation~\ref{eq:collectionlikelihood} into Baye's rule,
\begin{equation}
    \pr(
    \sptrees,
    \murates,
    \gfreqs
    \given
    \alldata
    )
    =
    \frac{
        \pr(
        \alldata
        \given
        \sptrees, \murates, \gfreqs
        )
        \pr(
        \sptrees,
        \murates,
        \gfreqs
        )
    }{
        \pr(
        \alldata
        )
    }.
    \label{eq:collectionindependentbayesrule}
\end{equation}
By expanding out the species trees into their component parts, the divergence
times and effective population sizes, we get
\begin{equation}
    \pr(
    \comparisondivtimes,
    \collectionpopsizes,
    \murates,
    \gfreqs
    \given
    \alldata
    )
    =
    \frac{
        \pr(
        \alldata
        \given
        \comparisondivtimes,
        \collectionpopsizes,
        \murates,
        \gfreqs
        )
        \pr(\comparisondivtimes, \collectionpopsizes, \murates, \gfreqs)
    }{
        \pr(
        \alldata
        )
    }.
    \label{eq:collectionindependentbayesruleexpanded}
\end{equation}
However, this assumes all the species trees are diverging independently, not allowing us
to learn about shared divergence times.
I.e., this would be the same as calculating the posterior of each pair
seperately.
What we want to do is relax the assumption that the species trees are
independent, and allow them to share the same divergence times.
We want to estimate how many divergences occurred, and which pairs, if any,
shared divergences.

Let's use \divtimemodel to represent the divergence model: The number of
divergence times, \nevents, which can range from 1 to \ncomparisons, and the
mapping of the population pairs to these \nevents divergences.
We will separate out \divtimemodel into two components,
\begin{enumerate}
    \item the partitioning of the \ncomparisons population pairs to
        divergence events, which we will denote as \divtimesets, and
    \item the divergence times themselves,
        $\divtimes = \divtime[1], \ldots, \divtime[\nevents]$.
\end{enumerate}
We use the Dirichlet process as a prior on divergence models,
$\divtimemodel \sim \dirp(\basedistribution, \concentration)$,
where \basedistribution is the base distribution of the process and
\concentration is concentration parameter that controls how clustered the
process is.

Under the Dirichlet process prior, the posterior becomes
\begin{equation}
    \pr(
    \concentration,
    \divtimemodel,
    \collectionpopsizes,
    \murates,
    \gfreq
    \given
    \alldata
    )
    =
    \frac{
        \pr(
        \alldata
        \given
        \divtimemodel,
        \collectionpopsizes,
        \murates,
        \gfreqs
        )
        \pr(\divtimemodel \given \concentration)
        \pr(\concentration)
        \pr(\collectionpopsizes)
        \pr(\murates)
        \pr(\gfreqs)
    }{
        \pr(
        \alldata
        )
    }.
    \label{eq:bayesrule}
\end{equation}
By expanding out the divergence model (\divtimemodel) into the
unique divergence times (\divtimes) and the partitioning of the pairs of populations
to those times (\divtimesets), we get
\begin{equation}
    \pr(
    \concentration,
    \divtimes,
    \divtimesets,
    \collectionpopsizes,
    \murates,
    \gfreq
    \given
    \alldata
    )
    =
    \frac{
        \pr(
        \alldata
        \given
        \divtimes,
        \divtimesets,
        \collectionpopsizes,
        \murates,
        \gfreqs
        )
        \pr(\divtimes \given \divtimesets)
        \pr(\divtimesets \given \concentration)
        \pr(\concentration)
        \pr(\collectionpopsizes)
        \pr(\murates)
        \pr(\gfreqs)
    }{
        \pr(
        \alldata
        )
    }.
    \label{eq:bayesruleexpanded}
\end{equation}

\subsubsection{Priors}

\paragraph{Prior on divergence times}
Given the divergence model, we use a gamma distribution for the prior on the
divergence times,
$\divtime[i] \given \divtimesets \sim \textrm{Gamma}(\cdot, \cdot)$.
This is the base distribution (\basedistribution) of the Dirichlet process.

\paragraph{Prior on effective population sizes}
For the two descendant populations of each pair, we use a gamma distribution as
the prior on the effective population size.
For the root population, we use a gamma distribution on the effective
population size \emph{relative} to the mean size of the two descendant
populations.
For example, a value of one would mean the root population size is equal to 
$(\epopsize[\descendantpopindex{1}] + \epopsize[\descendantpopindex{2}]) / 2$.
The goal of this is to allow more informative priors on the root population size,
because we often have stronger prior expectations for the relative effective
size of the ancestral population than the absolute size.
This is important, because the effective size of the ancestral population is a
difficult nuisance parameter to estimate and is correlated with the divergence
time.
For example, if the divergence time is so old such that all the gene copies
of a locus coalesce within the descendant population, the locus
provides very little information about the size of the ancestral
population.
As a result, a larger ancestral population and more recent divergence will have
a very similar likelihood to a small ancestral population and an older
divergence.
Thus, placing more prior density on reasonable values of the ancestral
population size can help improve the precision of divergence time estimates.

\paragraph{Prior on mutation rates}
In the model presented above, for each population pair, the divergence time
(\divtime) and mutation rate (\murate[i]) are inextricably linked.
For a single pair of populations, if little is known about the mutation rate,
this is easy to solve by simply setting it to one ($\murate[1] = 1$) such that
time is in units of expected subsitutions per site and the effective population
sizes are scaled by \murate.
However, what about the second pair of populations we wish to compare the
divergence time to the first?
Because the species trees in our model are disconnected, we cannot learn about
the relative rates of mutation across the population pairs.
As a result, we need strong prior information about the relative rates of
mutation across population pairs for this model to work.

If the second pair of populations is closely related to our first, and shares a similar life history,
we could assume they share the same mutation rate, and set the mutation rate of the second pair
to one as well ($\murate[2] = 1$).
Alternatively, we could relax that assumption and put a prior on \murate[2].
However, this needs to be an strongly informative prior.
Placing a very weakly informative prior on \murate[2] would mean that we can no
longer estimate its divergence time relative to the first pair.
So, while it is possible to incorporate uncertainty in mutation rates, that
uncertainty is directly transfered to uncertainty in divergence times.

\paragraph{Prior on equilibrium state frequency}
Our method allows for a beta prior to be placed on the frequency of the green
allele for each pair of populations,
$\gfreq[i] \sim \textrm{Beta}(\cdot, \cdot)$.
However, if using SNP data, we advise fixing the frequency of the red and green
states be equal (i.e., $\gfreq = 0.5$).
The reason for this is that there is no natural way of recoding 4-state
nucleotides to two states, and so the relative transition rates, \rgmurate and
\grmurate, are not biologically meaningful.
There will always be arbirariness associated with how one decides to perform
this recoding, and unless $\gfreq = 0.5$, this arbitrariness will affect the
likelihood and results.
This makes the CTMC model a two-state analog of the ``JC69'' model
\citep{JC1969}.
However, if the genetic markers are naturally biallelic, the frequencies of the
two states can be estimated, making the model a two-state general
time-reversible model \citep{Tavare1986}.



\subsection{Simulation-based analyses}

\subsubsection{Validation analyses}

\subsubsection{Assessing effect of linkage violation}
The characters of most datasets being collected by high-throughput technologies
do not all evolve along independent gene trees.
Most consist of many putatively unlinked loci that each comprise sequences of
linked nucleotides.
For example, ``RADseq'' and ``sequence capture'' techniques generate thousands
of loci that are approximately 50--300 nucleotides in length.
This creates a question when using methods that assume each character is
independent:
Is it better to violate that assumption and use all the data, or throw
away much of the data to avoid the violation?

To adhere to the independence assumption, we could retain only a single site
per locus.
This results in a very large loss of data.
Furthermore, to try and maximize the informativeness of the retained sites,
most researchers retain only variable characters.
While this can be corrected for, it still results in the loss of a very
informative component of the data: The proportion of variable sites.
Before throwing away so much information, we should determine whether
it is in our best interest.
In other words, does keeping the data and violating the independence
assumption result in better or worse inferences than throwing out the 
data?

To address this question, 
we simulated datasets composed of loci of linked sites that were 100, 500, and
1000 sites long.
The characters for each locus were simulated along the same gene tree (i.e., no
intra-locus recombination).
Simulated datasets were analyzed with ecovolity in one of three ways:
(1) All characters were included,
(2) only variable characters were included,
and
(3) only a maximum of one variable character per locus was included.
Only the last option avoids violating the assumption of unlinked characters,
but throws out the most data.

We simulated datasets with a total of 100k and 500k characters.
Relative root population was Gamma(100, 0.01).


\subsubsection{Assessing effect of missing data}
Under the model described above, missing data should not create any bias.
Because each character is modeled as unlinked and evolving along
a coalescent gene tree, the identity of each gene copy does not matter.
Thus, missing data for any particular site simply means a smaller sample of
gene copies were sampled from the population for that character.

To confirm this behavior, we simulated datasets with different sampling
probabilities.
Specifically, we simulated datasets for which probability of sampling each gene
copy was 90\%, 75\%, or 50\%, which resulted in datasets with approximately
10\%, 25\%, or 50\% missing data.

\subsubsection{Assesssing the effect of biases in character-pattern acquisition}
When analyzing the empirical data (see below),
we observed large discrepencies in the estimated divergences depending on
whether or not the constant sites were removed from the analysis.
This was not observed in the analyses of simulated data, because correcting the
likelihood is appropriately corrrected for the missing constant sites (see
above).
This suggests that there is additional site-pattern acquisition biases in the
empirical data, which are not corrected for.
Such acquistion biases have been documented during the denovo assembly of
RADseq loci \citep{Harvey2015,Linck2017}.

The loss of rare alleles during the acquisition and assembly of the data could
explain the much larger divergences estimated from the empirical data when
constant sites are removed.
After the constant sites, the rare alleles are next in line to inform the model
that the population divergence was recent.
If these patterns are being lost and not accounted for in the likelihood
correction, this should create an upward bias in the divergence time estimates.

To explore whether data acquisition bias can explain the discrepancy we
observed for the \spp{Gekko} data, we simulated datasets where the
probability of sampling singleton site patterns (i.e., one gene copy is different from all the others)
was 80\%, 60\%, and 40\%.


\subsubsection{Comparison to ABC methods}
We wanted to compare the performance of the new method to the existing
approximate-likelihood Bayesian computation (ABC) methods msBayes
\citep{Huang2011}, and dpp-msbayes \citep{Oaks2014dpp}.
In order to do this, we had to simulate relatively small datasets
that the ABC methods could handle.
We simulated 200, 200-base-pair loci.
We selected the settings to try and make the models as similar
as possible to maximize the comparability of the results.

Each method was applied to 100 data sets simulated under its own model.
Thus, there were no model violations, except for the new method, for which the
assumption of unlinked characters was violated by the 200 base-pair loci.

\subsection{Empirical application}
Previous methods for esimating shared divergence times often over-cluster taxa
\citep{Oaks2012,Oaks2014reply} or have little information to update prior
expectations \citep{Oaks2014dpp}.
Thus, a good empirical test of the new method would be pairs of populations
that we expect diverged randomly from one another.
We collected RADseq data from four pairs of populations of \spp{Gekko} lizards.
Each pair of populations inhabit two different oceanic islands in the
Philippines that were never connected during lower sea levels of glacial
periods.
Because these islands were never connected, the divergence between the
populations of each pair is necessarily due to dispersal, the timing of which
should be idiosyncratic to each pair.

To further test the method, we placed 50\% of the prior probability on one
divergence event (i.e., all four pairs sharing the same divergence).


\section{Results}

\section{Discussion}
