\mFigure{../../simulations/validation/plots/div-time-scatter.pdf}{
    The accuracy and precision of divergence time estimates, in units of
    expected subsitutions per site, when data are simulated and analyzed
    under the same model
    (i.e., no model misspecification).
    \validationfiguregriddescription
    Each plotted circle and associated error bars represent the posterior mean
    and 95\% credibility interval for the time that a pair of populations
    diverged.
    Each plot consists of 1500 estimates---500 simulated \datasets, each with
    three pairs of populations.
    For each plot, the root-mean-square error (RMSE) and the proportion of
    divergence-time estimates for which the 95\% posterior credibility interval
    contained the true divergence time---$p(\comparisondivtime \in
    \textrm{CI})$---is given.
    Plot generated with matplotlib Version 2.0.0 \citep{matplotlib}.
}{fig:valdivtimes}

\mFigure{../../simulations/validation/plots/100k-sites-est-vs-true-prob-nevent-1.pdf}{
    Assessing frequentist behavior of divergence-model posterior probabilities
    when there is no model misspecification.
    20,500 \datasets were simulated and analyzed under the same model and
    assigned to bins of width 0.2 based on the estimated posterior
    probability of a single, shared divergence event.
    The mean posterior probability of each bin is plotted agains the proportion
    of \datasets in the bin for which a single, shared divergence is the
    true model.
    The number of \datasets within each bin is provided next to the
    corresponding plotted point.
    The left plot shows the results when all characters are analyzed, and the
    right plot shows the results when only the variable characters are
    analyzed.
    Plot generated with matplotlib Version 2.0.0 \citep{matplotlib}.
}{fig:valpostprobs}

\mFigure{../../simulations/validation/plots/nevents.pdf}{
    The ability of the new method to estimate the number of divergence events
    when data are simulated and analyzed
    under the same model
    (i.e., no model misspecification).
    \validationfiguregriddescription
    Each plot shows the results of the analyses of 500 simulated \datasets;
    the number of \datasets that fall within each possible cell
    of true versus estimated numbers of events is shown, and cells with
    more \datasets are shaded darker.
    For each plot,
    the proportion of \datasets for which the number of events with the largest
    posterior probability matched the true number of events---$p(\hat{\nevents}
    = \nevents)$---is shown in the upper left corner,
    the median posterior probability of the correct number of events across all
    \datasets---$\widetilde{p(\nevents)}$---is shown in the upper right corner,
    and
    the proportion of \datasets for which the true divergence model was
    included in the 95\% posterior credibility set---$p(\nevents \in 95\%
    \textrm{CS})$---is shown in the lower right.
    Plot generated with matplotlib Version 2.0.0 \citep{matplotlib}.
}{fig:valnevents}


\mFigure{../../simulations/validation/plots/linkage-500k-div-time-scatter.pdf}{
    Assessing the affect of linked sites on the
    the accuracy and precision of divergence time estimates (in units of
    expected subsitutions per site).
    The columns, from left to right, show the results when loci are simulated
    to comprise 100, 500, and 1000 linked sites.
    For each simulated \dataset, each of three population pairs has
    500,000 sites total.
    The rows show the results when (top) all sites, (middle) all variable
    sites, and (bottom) at most one variable site per locus are analyzed.
    For each plot, the root-mean-square error (RMSE) and the proportion of
    divergence-time estimates for which the 95\% posterior credibility interval
    contained the true divergence time---$p(\comparisondivtime \in
    \textrm{CI})$---is given.
    Plot generated with matplotlib Version 2.0.0 \citep{matplotlib}.
}{fig:linkagedivtimes500k}

\mFigure{../../simulations/validation/plots/linkage-500k-nevents.pdf}{
    Assessing the affect of linked sites on the ability of the new method to
    estimate the number of divergence events.
    The columns, from left to right, show the results when loci are simulated
    to comprise 100, 500, and 1000 linked sites.
    For each simulated \dataset, each of three population pairs has
    500,000 sites total.
    The rows show the results when (top) all sites, (middle) all variable
    sites, and (bottom) at most one variable site per locus are analyzed.
    For each plot,
    the proportion of \datasets for which the number of events with the largest
    posterior probability matched the true number of events---$p(\hat{\nevents}
    = \nevents)$---is shown in the upper left corner,
    the median posterior probability of the correct number of events across all
    \datasets---$\widetilde{p(\nevents)}$---is shown in the upper right corner,
    and
    the proportion of \datasets for which the true divergence model was
    included in the 95\% posterior credibility set---$p(\nevents \in 95\%
    \textrm{CS})$---is shown in the lower right.
    Plot generated with matplotlib Version 2.0.0 \citep{matplotlib}.
}{fig:linkagenevents500k}

\mFigure{../../simulations/validation/plots/linkage-100-100k-sites-est-vs-true-prob-nevent-1.pdf}{
    Assessing the affect of linked sites on the frequentist behavior of
    divergence-model posterior probabilities.
    10,500 \datasets were simulated such that each of 3 population pairs has
    1000 loci, each with 100 linked sites (100,000 sites total).
    Each simulated \dataset is assigned to a bin of width 0.2 based on the
    estimated posterior probability of a single, shared divergence event.
    The mean posterior probability of each bin is plotted agains the proportion
    of \datasets in the bin for which a single, shared divergence is the
    true model.
    The number of \datasets within each bin is provided next to the
    corresponding plotted point.
    The plots show the results when (left) all characters, (middle) all
    variable characters, and (right) at most one variable character per locus
    is analyzed.
    Plot generated with matplotlib Version 2.0.0 \citep{matplotlib}.
}{fig:linkagepostprobs}


\mFigure{../../simulations/validation/plots/missing-data-div-time-scatter.pdf}{
    Assessing the affect of missing data on the
    the accuracy and precision of divergence time estimates (in units of
    expected subsitutions per site).
    The columns, from left to right, show the results when each simulated
    500,000-character matrix has approximately 0\%, 10\%, 25\%, and 50\%
    missing cells.
    The rows show the results when (top) all sites and (bottom) only variable
    sites are analyzed.
    For each plot, the root-mean-square error (RMSE) and the proportion of
    divergence-time estimates for which the 95\% posterior credibility interval
    contained the true divergence time---$p(\comparisondivtime \in
    \textrm{CI})$---is given.
    Plot generated with matplotlib Version 2.0.0 \citep{matplotlib}.
}{fig:missingdivtimes}

\mFigure{../../simulations/validation/plots/missing-data-nevents.pdf}{
    Assessing the affect of missing data on the ability of the new method to
    estimate the number of divergence events.
    The columns, from left to right, show the results when each simulated
    500,000-character matrix has approximately 0\%, 10\%, 25\%, and 50\%
    missing cells.
    The rows show the results when (top) all sites and (bottom) only variable
    sites are analyzed.
    For each plot,
    the proportion of \datasets for which the number of events with the largest
    posterior probability matched the true number of events---$p(\hat{\nevents}
    = \nevents)$---is shown in the upper left corner,
    the median posterior probability of the correct number of events across all
    \datasets---$\widetilde{p(\nevents)}$---is shown in the upper right corner,
    and
    the proportion of \datasets for which the true divergence model was
    included in the 95\% posterior credibility set---$p(\nevents \in 95\%
    \textrm{CS})$---is shown in the lower right.
    Plot generated with matplotlib Version 2.0.0 \citep{matplotlib}.
}{fig:missingnevents}


\mFigure{../../simulations/validation/plots/filtered-data-div-time-scatter.pdf}{
    Assessing the affect of an acquisition bias against rare allele patterns on
    the accuracy and precision of divergence time estimates (in units of
    expected subsitutions per site).
    The columns, from left to right, show the results when each simulated
    500,000-character \dataset has a probability of 100\%, 80\%, 60\%, and 40\%
    of sampling each simulated singleton pattern.
    I.e., the \datasets analyzed in the far right column is missing
    approximately 60\% of characters where all but one gene copy has the same
    allele.
    The rows show the results when (top) all sites and (bottom) only variable
    sites are analyzed.
    For each plot, the root-mean-square error (RMSE) and the proportion of
    divergence-time estimates for which the 95\% posterior credibility interval
    contained the true divergence time---$p(\comparisondivtime \in
    \textrm{CI})$---is given.
    Plot generated with matplotlib Version 2.0.0 \citep{matplotlib}.
}{fig:filtereddivtimes}

\mFigure{../../simulations/validation/plots/filtered-data-nevents.pdf}{
    Assessing the affect of an acquisition bias against rare allele patterns
    on the ability of the new method to estimate the number of divergence
    events.
    The columns, from left to right, show the results when each simulated
    500,000-character \dataset has a probability of 100\%, 80\%, 60\%, and 40\%
    of sampling each simulated singleton pattern.
    I.e., the \datasets analyzed in the far right column is missing
    approximately 60\% of characters where all but one gene copy has the same
    allele.
    The rows show the results when (top) all sites and (bottom) only variable
    sites are analyzed.
    For each plot,
    the proportion of \datasets for which the number of events with the largest
    posterior probability matched the true number of events---$p(\hat{\nevents}
    = \nevents)$---is shown in the upper left corner,
    the median posterior probability of the correct number of events across all
    \datasets---$\widetilde{p(\nevents)}$---is shown in the upper right corner,
    and
    the proportion of \datasets for which the true divergence model was
    included in the 95\% posterior credibility set---$p(\nevents \in 95\%
    \textrm{CS})$---is shown in the lower right.
    Plot generated with matplotlib Version 2.0.0 \citep{matplotlib}.
}{fig:filterednevents}


\mFigure{../../bake-off/plots/divergence-time-scatter.pdf}{
    Comparing the accuracy and precision of divergence-time estimates
    between (left) the new full-likelihood Bayesian method, \ecoevolity, and
    (right) the approximate-likelihood Bayesianc computation method,
    \dppmsbayes.
    Each plotted circle and associated error bars represent the posterior mean
    and 95\% credibility interval for the time that a pair of populations
    diverged.
    Each plot consists of 1500 estimates---500 simulated \datasets, each with
    three pairs of populations.
    The simulated character matrix for each population pair consisted of 200 loci,
    each with 200 linked sites (40,000 characters total).
    For each plot, the root-mean-square error (RMSE) and the proportion of
    divergence-time estimates for which the 95\% posterior credibility interval
    contained the true divergence time---$p(\comparisondivtime \in
    \textrm{CI})$---is given.
    Plot generated with matplotlib Version 2.0.0 \citep{matplotlib}.
}{fig:bakeoffdivtimes}

\mFigure{../../bake-off/plots/nevents.pdf}{
    Comparing the ability to estimate the number of divergence events between
    (left) the new full-likelihood Bayesian method, \ecoevolity, and (right)
    the approximate-likelihood Bayesianc computation method, \dppmsbayes.
    Each plot shows the results of the analyses of 500 simulated \datasets;
    the number of \datasets that fall within each possible cell
    of true versus estimated numbers of events is shown, and cells with
    more \datasets are shaded darker.
    Each simulated \dataset contained three pairs of populations, and the
    simulated character matrix for each pair consisted of 200 loci, each with
    200 linked sites (40,000 characters total).
    For each plot,
    the proportion of \datasets for which the number of events with the largest
    posterior probability matched the true number of events---$p(\hat{\nevents}
    = \nevents)$---is shown in the upper left corner,
    the median posterior probability of the correct number of events across all
    \datasets---$\widetilde{p(\nevents)}$---is shown in the upper right corner,
    and
    the proportion of \datasets for which the true divergence model was
    included in the 95\% posterior credibility set---$p(\nevents \in 95\%
    \textrm{CS})$---is shown in the lower right.
    Plot generated with matplotlib Version 2.0.0 \citep{matplotlib}.
}{fig:bakeoffnevents}


\mFigure{../../gekko-results/sumevents.pdf}{
    The prior (light bars) and approximated posterior (dark bars) probabilities
    of the number of divergence events across \spp{Gekko} pairs of populations,
    under eight different combinations of priors on the divergence times (rows)
    and the concentration parameter of the Dirichlet process (columns).
    The Bayes factor for each number of divergence times is given above the
    corresponding bars.
    Each Bayes factor compares the given number of events 
    to all other possible numbers of divergence events.
    Plots generated with ggplot2 Version 2.2.1 \citep{ggplot2}.
}{fig:gekkonevents}

\mFigure{../../gekko-results/sumtimes.pdf}{
    The approximate marginal posterior densities of divergence times for each
    \spp{Gekko} pair of populations,
    under eight different combinations of priors on the divergence times (rows)
    and the concentration parameter of the Dirichlet process (columns).
    Plots generated with ggridges Version 0.4.1 \citep{ggridges041}
    and ggplot2 Version 2.2.1 \citep{ggplot2}.
}{fig:gekkodivtimes}
